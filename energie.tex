 \chapter{Energie}
Die Energie ist eine fundamentale Grösse aller Teilbereiche der Physik.
Diese kann in verschiedenen Formen auftreten. In der Mechanik werden vor
allem die potentielle und kinetische Energie behandelt.

\section{Definition}

\section{Energiearten}

\subsection{Potentielle Energie}
Die potentielle Energie ist definiert als jene Energie, welche sich aus
einem Höhenunterschied (Potentialdifferenz) einer Masse zu einer 
Referenzhöhe ergibt. 
\[ \boxed{ E_{Pot} = m \cdot g \cdot (h - h_{Ref}) } \]

\subsection{Kinetische Energie}
Die kinetische Energie ist proportional zur Masse des sich bewegenden Körpers
und quadratisch zu deren Geschwindigkeit. Wichtig ist hierbei auch die 
Betrachtung der Bewegung des Körpers, vor allem zu welcher Referez die 
Geschwindigkeit gilt (Relativgeschwindugkeit).
\[ \boxed{ E_{Kin_{Translation}} = \frac{m \cdot \vec{v}^2}{2} } \]
Die kinetische Energie kann aber auch eine Rotation beschreiben.
\[ \boxed{ E_{Kin_{Rotation}} = \frac{\vec{J} \cdot \omega^2}{2} } \]
\subsection{Federenergie}
Die Federengergie ist definiert als das Produkt aus Federkonstante $k$ und 
dem Integral der Dehnung bzw. des Weges $s$ 
(siehe Kapitel \ref{sec:feder-energie}).
\[ \boxed{E_{Feder} = k \cdot \int_{s_a}^{s_b} s \cdot ds = \frac{k \cdot s^2}{2}} \]
