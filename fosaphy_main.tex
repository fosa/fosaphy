% coding:utf-8

%----------------------------------------
%FOSAPHY, a LaTeX-Code for a summary of basic physics
%Copyright (C) 2013, Daniel Winz, Ervin Mazlagic

%This program is free software; you can redistribute it and/or
%modify it under the terms of the GNU General Public License
%as published by the Free Software Foundation; either version 2
%of the License, or (at your option) any later version.

%This program is distributed in the hope that it will be useful,
%but WITHOUT ANY WARRANTY; without even the implied warranty of
%MERCHANTABILITY or FITNESS FOR A PARTICULAR PURPOSE.  See the
%GNU General Public License for more details.
%----------------------------------------

\documentclass[a5paper,10pt,fleqn]{book}

\usepackage{fosaphy_layout}

%\setboolean{ti}{true}        % Anleitung für TI-89
%\setboolean{nspire}{true}    % Anleitung für TI-Nspire CAS
%\setboolean{tiboth}{true}    % Anleitung für beide TR (Nur für Titelseite relevant)

\title{Formelsammlung Physik\\\textcolor{red}{\textbf{\Huge{Achtung! \\Diese Formelsammlung wird bis zur Modul\-end\-prüfung nicht mehr erweitert. Wir empfehlen für die Modulendprüfung das Taschenbuch Physik von Horst Kuchling oder die Formelsammlung von Aurel Hunkeler. }}}}
% \ifti
% \title{Formelsammlung Physik \\ Mit Anleitung zu TI-89}
% \fi
% \ifnspire
% \title{Formelsammlung Physik \\ Mit Anleitung zu TI-Nspire}
% \fi
% \iftiboth
% \title{Formelsammlung Physik \\ Mit Anleitung zu TI-89 und TI-Nspire CAS}
% \fi

\author{Daniel Winz\\Ervin Mazlagi\'c\\Mario Felder}
\date{\today~\dtc}

\begin{document}

\maketitle

% coding:utf-8

%----------------------------------------
%FOSAPHY, a LaTeX-Code for a summary of basic physics
%Copyright (C) 2013, Daniel Winz, Ervin Mazlagic

%This program is free software; you can redistribute it and/or
%modify it under the terms of the GNU General Public License
%as published by the Free Software Foundation; either version 2
%of the License, or (at your option) any later version.

%This program is distributed in the hope that it will be useful,
%but WITHOUT ANY WARRANTY; without even the implied warranty of
%MERCHANTABILITY or FITNESS FOR A PARTICULAR PURPOSE.  See the
%GNU General Public License for more details.
%----------------------------------------

\chapter*{Über diese Arbeit}
Dies ist das Ergebnis einer Zusammenarbeit auf Basis freier Texte erstellt von Studierenden der Fachhochschule Luzern und ist unter der GPLv2 lizenziert. Der \TeX - bzw. \LaTeX -Code ist auf \url{github.com/daniw/fosaet} hinterlegt.Eine aktuelle PDF-Ausgabe steht auf \url{fosa.adinox.ch} zum Download bereit.

In dieser Formelsammlung sind die Inhalte des Physikteil des Moduls Ma+PHY1T der HSLU-T\&A zusammengefasst. 


\tableofcontents

% coding:utf-8

%----------------------------------------
%FOSAPHY, a LaTeX-Code for a summary of basic physics
%Copyright (C) 2013, Daniel Winz, Ervin Mazlagic

%This program is free software; you can redistribute it and/or
%modify it under the terms of the GNU General Public License
%as published by the Free Software Foundation; either version 2
%of the License, or (at your option) any later version.

%This program is distributed in the hope that it will be useful,
%but WITHOUT ANY WARRANTY; without even the implied warranty of
%MERCHANTABILITY or FITNESS FOR A PARTICULAR PURPOSE.  See the
%GNU General Public License for more details.
%----------------------------------------

% coding:utf-8

%----------------------------------------
%FOSAPHY, a LaTeX-Code for a summary of basic physics
%Copyright (C) 2013, Daniel Winz, Ervin Mazlagic

%This program is free software; you can redistribute it and/or
%modify it under the terms of the GNU General Public License
%as published by the Free Software Foundation; either version 2
%of the License, or (at your option) any later version.

%This program is distributed in the hope that it will be useful,
%but WITHOUT ANY WARRANTY; without even the implied warranty of
%MERCHANTABILITY or FITNESS FOR A PARTICULAR PURPOSE.  See the
%GNU General Public License for more details.
%----------------------------------------

\section{SI Einheiten}
\subsection{Grundeinheiten}
\begin{tabular}{llll}
  \rowcolor{white} \textbf{Basisgrösse} & \textbf{Symbol} 
                    & \textbf{Einheit} & \textbf{Zeichen}\\
  \rowcolor{lgray} Länge       & $l$   & Meter     & $m$\\
  \rowcolor{white} Zeit        & $t$   & Sekunde   & $s$\\
  \rowcolor{lgray} Masse       & $m$   & Kilogramm & $kg$\\
  \rowcolor{white} Temperatur  & $T$   & Kelvin    & $K$\\
  \rowcolor{lgray} Stromstärke & $I$   & Ampere    & $A$\\
  \rowcolor{white} Stoffmenge  & $n$   & Mol       & $mol$\\
  \rowcolor{lgray} Lichtstärke & $I_v$ & Candela   & $cd$\\
\end{tabular}

% \subsection{abgeleitete Einheiten}
% \begin{tabular}{lll}
%   \rowcolor{white} \textbf{}
% \end{tabular}          		% SI Einheiten
\chapter{Newtonsche Axiome}
Die Newtonschen Axiome bilden das Fundament der klassischen Mechanik. Diese
hatte Newton im Jahre 1687 in seinem Werk 
\textit{Philosophiae Naturalis Principia Mathematica} formuliert. 

\section{Erstes Newtonsches Axiom}
\textit{"`Ein Körper verharrt im Zustand der Ruhe oder der gleichförmigen 
Translation, sofern er nicht durch einwirkende Kräfte zur Änderung seines 
Zustandes gezwungen wird."' --- Trägheit}

\[ \boxed{\vec{v} = \text{konstant}} \quad \text{falls } \sum\vec{F}=0 \]

\section{Zweites Newtonsches Axiom} 
\textit{"`Die Änderung der Bewegung ist der Einwirkung der bewegenden Kraft 
proportional und geschieht nach der Richtung derjenigen geraden Linie, nach
welcher jene Kraft wirkt."' --- Kraft}

\[ \boxed{\vec{F} = m \cdot \vec{a}} \]

\section{Drittes Newtonsches Axiom}
\textit{"`Kräfte treten immer paarweise auf. Übt ein Körper A auf einen 
anderen Körper B eine Kraft aus (actio), so wirkt eine gleich grosse, aber
entgegen gerichtete Kraft von Körper B auf Körper A (reactio)."' --- 
Aktion und Reaktion}

\[ \boxed{\vec{F}_{A \rightarrow B} = - \vec{F}_{B \rightarrow A}} \]
	% newtonsche Axiome
% coding:utf-8

%----------------------------------------
%FOSAPHY, a LaTeX-Code for a summary of basic physics
%Copyright (C) 2013, Mario Felder

%This program is free software; you can redistribute it and/or
%modify it under the terms of the GNU General Public License
%as published by the Free Software Foundation; either version 2
%of the License, or (at your option) any later version.

%This program is distributed in the hope that it will be useful,
%but WITHOUT ANY WARRANTY; without even the implied warranty of
%MERCHANTABILITY or FITNESS FOR A PARTICULAR PURPOSE.  See the
%GNU General Public License for more details.
%----------------------------------------

\chapter{Bewegung}

\newpage
\section{Gerade Bewegung}
Die gerade Bewegung kennt vier elementare Grössen: 
Weg $\vec{s}$, Geschwindigkeit $\vec{v}$, Beschleunigung $\vec{a}$ und
die Zeit $t$. Alle diese Grössen, mit Ausnahme der Zeit $t$, sind 
vektorielle Grössen.

Diese drei vektoriellen Grössen, welche alle Funktionen der Zeit
sind, können gegenseiten jeweils durch differenzieren oder integrieren 
nach der Zeit $t$ hergeleitet werden, denn es gilt der folgende 
Zusammenhang.

\[ \boxed{\begin{array}{c c c c c}
	\vec{x}(t) 
		& \xrightarrow{\frac{d}{dt}}
	& \vec{v}(t)
		& \xrightarrow{\frac{d}{dt}}
	& \vec{a}(t) \\
	& & & & \\
	\vec{x}(t) 
		& \xleftarrow{\int dt}
	& \vec{v}(t)
		& \xleftarrow{\int dt}
	& \vec{a}(t) \\
\end{array}} \]

\noindent
Die Definitionen für die Momentanwerte sind dabei die Folgenden.
\[\boxed{\begin{array}{r l}
	\vec{v}	&
		= \displaystyle \lim\limits_{P_2 \rightarrow P_1} 
			{\left( \frac{x_2 - x_1}{t_2 - t_1} \right)}
		= \displaystyle \lim\limits_{\Delta t \rightarrow 0}
			{\frac{\Delta x}{\Delta t}}
		= \displaystyle \frac{\mathrm{d}x}{\mathrm{d}t}=\dot{x} \\
	& \\
	\vec{a} &
		= \displaystyle \lim\limits_{P_2 \rightarrow P_1}
			{\left( \frac{\vec{v}_2 
			- \vec{v}_1}{t_2 - t_1} \right)}
		= \displaystyle \lim\limits_{\Delta t \rightarrow 0}
			{\frac{\Delta v}{\Delta t}}
		= \displaystyle \frac{\mathrm{d}v}{\mathrm{d}t}
		= \displaystyle \dot{v}
		= \ddot{x} \\
	& \\
	\Delta \vec{s} &
		= \displaystyle \lim\limits_{n \rightarrow \infty}
			{\left( \sum_{i=1}^{n} \Delta \vec{s}_i \right)}
		= \displaystyle \lim\limits_{\Delta t_i \rightarrow 0}
			{\left(
				\sum^{n}_{i=1}\vec{v}_i\cdot\Delta t
			\right)}
		= \displaystyle \int_{t_A}^{t_B}\vec{v}\mathrm{d}t
\end{array}}\]

\section{Bewegung mit konstanter Beschleunigung}
Bei einer Bewegung mit konstanter Beschleunigung $\vec{a}$, wie dies beim
freien Fall oder dem schiefen Wurf der Fall ist, gelten die folgenden
Zusammenhänge zwischen Weg $\vec{s}$, Geschwindigkeit $\vec{v}$ und 
Zeit $t$.

\begin{figure}[h!]
	\centering
	\includegraphics[scale=0.7]{bewegung.pdf}
	\caption{Gleichmässig beschleunigte Translation.}
	\label{fig:bewegung}
\end{figure}

\noindent
Mit den in der Grafik \ref{fig:bewegung} gezeigten Zusammenhängen lassen
sich Geschwindigkeit $\vec{v}$ und Weg $\vec{s}$ wie folgt beschreiben.
\[ \boxed{\begin{array}{r l} 
	\vec{a}(t) 	&= \displaystyle
		\vec{a} \\
	& \\
	\vec{v}(t)	&= \displaystyle 
		\vec{v}_0+\vec{a}\cdot t \\
	& \\
	\vec{s}(t) 	&= \displaystyle 
		\vec{s}_0+\vec{v}_0\cdot t+\frac{1}{2}\cdot\vec{a}\cdot t^2
\end{array}}\]
Bei einer solchen Bewegung mit konstanter Beschleunigung $\vec{a}$ gelten
insbesondere auch die folgenden Vereinfachungen.
\[ \boxed{\begin{array}{r l}
	\Delta \vec{s}	&= \displaystyle
		\vec{s}-\vec{s}_0 = \displaystyle
		\vec{v}_0\cdot t+\frac{1}{2}\cdot \vec{a}\cdot t^2 \\
	& \\
	\vec{v}^2 	&= \displaystyle
		\vec{v}_{0}^{2}+2\vec{a} \cdot \Delta \vec{s} \\
	& \\
	\vec{v}_{av} &= \displaystyle
		\frac{1}{2} \left( \vec{v}_0 + \vec{v} \right) \\
	& \\
	\vec{a} 	&= \displaystyle
		\frac{\vec{v}(t)^2-\vec{v}_0^2}{2\cdot\Delta \vec{s}} 
\end{array}}\]
Mit den obigen Vereinfachungen lassen sich nun die Formeln für 
$\vec{a}$, $\vec{v}$, $\vec{s}$ und $t$ durch je drei
Kombinationen formulieren (Winz'sche Tabelle).
\[ \boxed{\begin{array}{c c c c c c c}
	\vec{s}	
		&=& \displaystyle 
			\frac{1}{2} \cdot \vec{v} \cdot t
		&=& \displaystyle 
			\frac{1}{2} \cdot \frac{\vec{v}^2}{\vec{a}}
		&=& \displaystyle 
			\frac{1}{2} \cdot \vec{a} \cdot t^2 \\
	& & & & & & \\
	\vec{v} 
		&=& \displaystyle 
			\frac{1}{2} \cdot \frac{\vec{s}}{t}
		&=& \displaystyle 
			\sqrt{2 \cdot \vec{a} \cdot \vec{s}}
		&=& \displaystyle 
			\vec{a} \cdot t \\
	& & & & & & \\
	\vec{a}
		&=& \displaystyle 
			\frac{\vec{v}}{t} 
		&=& \displaystyle 
			\frac{1}{2} \cdot \frac{\vec{v}^2}{\vec{s}}
		&=& \displaystyle 
			2 \cdot \frac{\vec{s}}{t^2} \\
	& & & & & & \\
	t 	
		&=& \displaystyle 
			\frac{\vec{v}}{\vec{a}}
		&=& \displaystyle 
			\sqrt{\frac{2 \cdot \vec{s}}{\vec{a}}}
		&=& \displaystyle 
			2 \cdot \frac{\vec{s}}{\vec{v}}
\end{array}}\]

\section{Bewegung im Raum}
Da die Grössen der Bewegung vektoriell sind, können diese auch 
komponentenweise betrachtet werden.
\[ \boxed{\begin{array}{r l}
	\vec{\Delta r} &
		= \vec{r}_2-\vec{r}_1
		= (x_2-x_1,y_2-y_1,z_2-z_1) \\
	& \\
	v \rightarrow \vec{v} &
		= \lim\limits_{\Delta t \rightarrow 0}
			{\frac{\Delta \vec{r}}{\Delta t}}
		= \frac{d\vec{r}}{dt}
		= \left(\frac{dx}{dt},\frac{dy}{dt},\frac{dz}{dt}\right) \\
	& \\
	a \rightarrow \vec{a} &
		= \lim\limits_{\Delta t \rightarrow 0}
			{\frac{\Delta \vec{v}}{\Delta t}}
		= \frac{d\vec{v}}{dt}
		= \frac{d^2\vec{r}}{dt^2}
\end{array}}\]
		
\section{Bahnkurve}

\begin{figure}[h!]
	\centering
	\includegraphics[scale=0.8]{bahnkurve.pdf}
	\caption{Bahnkurve mit Tangente und Normale dazu.}
	\label{fig:bahnkurve}
\end{figure}

\noindent
Betrachtet man die Grössen der Bewegung als Vektoren, so können diese 
in ihre jeweiligen Komponenten in $x,y,z$ zerlegt werden. Hierbei gilt 
insbesondere für die Geschwindigkeit $\vec{v}$ und die Beschleunigung
$\vec{a}$, dass diese entlang einer Bahnkurve stets normal zueinander 
sind oder anders formuliert:

\begin{itemize}
	\item Die Geschwindigkeit $\vec{v}$ liegt immer tangential an
		der Bahnkurve an.
	\item Gie Beschleunigung $\vec{a}$ zeigt immer nach innen, 
		d.h. normal zur Geschwindigkeit $\vec{v}$.
\end{itemize}

\section{Schiefer Wurf}
Der schiefe Wurf bezeichnet eine überlagerte Bewegung in mindestens zwei
Richtungen (z.B. $x$ und $y$). Diese Bewegung kann komponentenweise 
analysiert werden und per Superposition wieder zusammengesetzt werden.

\[\boxed{\begin{array}{r l  r l}
	& \vec{x}\text{-Komponente} & & \vec{y}\text{-Komponente} \\
	& & & \\
	\vec{a}_x 
		& = 0 
		& \qquad \vec{a}_y 
		& = -\vec{g} \\
	 & & &  \\
	\vec{v}_x 
		& = \vec{v}_0 \cdot cos(\alpha_0)
		& \vec{v}_y
		& = \vec{v}_0 \cdot sin(\alpha_0) -\vec{g} \cdot t \\
	 & & & \\
	\vec{s}_x
		& = \left( \vec{v}_0 \cdot cos(\alpha_0) \right) \cdot t
		& \vec{s}_y
		& = \left( \vec{v}_0 \cdot sin(\alpha_0) \right) \cdot t 
			- \frac{1}{2}\vec{g} \cdot t^2
\end{array}}\]

\noindent
Aus den obigen Zusammenhängen zum schiefen Wurf lässt sich eine 
Bahngleichung aufstellen, welche die Funktionen von $\vec{x}(t)$ und
$\vec{y}(t)$ kombiniert zu einer Funktion $\vec{y}(\vec{x})$.

\begin{figure}[h!]
	\centering
	\includegraphics[scale=0.8]{wurf.pdf}
	\caption{Schiefer Wurf.}
	\label{fig:wurf}
\end{figure}

\[ \boxed{
	\vec{y} = \vec{x} \cdot tan(\alpha) 
		- \frac{\vec{g}}{2 \cdot \vec{v}_0^2 \cdot cos^2(\alpha)}
		\cdot \vec{x}^2
} \]

\noindent
Wichtig ist hierbei zu beachten, dass die passende Referenz für $\vec{y}$
gewählt wird, denn es gelten dann die folgenden Bedingungen.

\[ \boxed{\begin{array}{l l}
	\text{Landestelle tiefer als Abwurfstelle} & \Rightarrow y < 0 \\
	& \\
	\text{Landestelle gleich wie Abwurfstelle} & \Rightarrow y = 0 \\
	& \\
	\text{Landestelle höher als Abwurfstelle} & \Rightarrow y > 0 
\end{array}} \]


\section{Schräge Zerlegung}
Die Komponentenzerlegung eines Vektors kann beliebig erfolgen. 
D.h. die Zerlegung muss nicht zwingend in $x,y$ oder $z$ erfolgen,
sondern beliebig im Raum. Bei einigen Bewegungen ist dies von Vorteil,
etwa beim waagrechten Wurf (Spezialfall des schiefen Wurfs).

\begin{figure}[h!]
	\centering
	\includegraphics[scale=0.8]{wurf2.pdf}
	\caption{Schräge Komponentenzerlegung am waagrechten Wurf.}
	\label{fig:wurf2}
\end{figure}

\noindent
Mit dieser Zerlegung können die folgenden Zusammenhänge formuliert werden.
\[ \boxed{\begin{array}{r l}
	\vec{s} &
		= \vec{v}_0 \cdot t
		= \vec{v}_0 \cdot \sqrt{\frac{2 \cdot \vec{y}}{\vec{g}}} \\
	& \\
	\vec{y} &
		= \frac{1}{2} \cdot \vec{g} \cdot t^2 \\
	& \\
	\vec{v}_B &
		= \vec{v}_0 + \vec{g} \cdot t 
\end{array}}\]

    		% Bewegung
\chapter{Schiefe Ebene}

Die schiefe Ebene stellt ein klassisches Problem der Mechanik. 
Diese kombiniert einfache Kräftebetrachtungen und zeigt den 
vektoriellen Charakter von Kräften und Beschleunigungen auf, welche sich 
mit der Anwendung trigonometrischer Gesetze bestimmen lassen.

\newpage
\section{Kräfte am Hang}
\begin{figure}[h!]
	\centering
	\includegraphics[scale=1]{../fig/schiefe-ebene-1.pdf}
	\caption{Kräftesystem mit zwei Massen am Hang}
	\label{fig:hangsystem}
\end{figure}

\subsection{Gewichtskraft}
Die Gewichtskraft gilt am Hang genau gleich wie auf der waagerechten Ebene.
\[ \boxed{\vec{F}_G = m \cdot \vec{g}} 
	\qquad \vec{g} = \vec{a}_{Erde} \approx 9.81\frac{m}{s^2} \]

\subsection{Normalkraft}
Die Normalkraft ist jene Kraftkomponente, welche von der Unterlage senkrecht
durch den Schwerpunkt der Masse geht, welche sie trägt. Somit ist die 
Normalkraft eine Funktion des Neigungswinkels und der Gewichtskraft, d.h.
$\vec{F}_N = f(\vec{F}_G, \alpha)$.
\[ \boxed{ \vec{F}_N 
	= \vec{F}_G \cdot cos(\alpha) 
	= m \cdot \vec{g} \cdot cos(\alpha)} 
\]
\[ \begin{array}{ll}
	\alpha \rightarrow 0 & \Rightarrow \vec{F}_N = \vec{F}_G \\
	\alpha \rightarrow \pi & \Rightarrow \vec{F}_N = 0
\end{array} \]

\subsection{Hangkraft}
Die Hangkraft ist jene Kraftkomponente, welche parallel zur Unterlage in 
Richtung der Gravitation zeigt. Die Wirkungslinie geht dabei wie bei der 
Normalkraft duch den Schwerpunkt der Masse und ist ebenfalls eine Funktion
des Neigungswinkels und der Gewichtskraft, d.h. 
$\vec{F}_N = f(\vec{F}_G, \alpha)$.
\[ \boxed{ \vec{F}_H  
	= \vec{F}_G \cdot sin(\alpha) 
	= m \cdot \vec{g} \cdot sin(\alpha)} 
\]
\[ \begin{array}{ll}
	\alpha \rightarrow 0 & \Rightarrow \vec{F}_N = 0 \\
	\alpha \rightarrow \pi & \Rightarrow \vec{F}_N = \vec{F}_G
\end{array} \]

\subsection{Seilkraft}
Die Seilkraft welche zwischen zwei miteinander verbundenen Massen auftritt
ist überall im Seil die selbe und zieht mit gleichem Betrag an beiden Massen.
\[ \vec{F}_S = |\vec{F}_{S_1}| = |\vec{F}_{S_2}| \]
Die Summe der Seilkräfte (es gibt maximal zwei) muss also folglich Null ergeben.
\[ \sum_{i=1}^{n=2} \vec{F}_{S_i} = \vec{F}_{S_1} + \vec{F}_{S_2} = 0 \]

\subsection{Reibungskraft}
Die Reibungskraft wirkt stets entgegen der Bewegungsrichtung (bremsend) und ist
definiert als das Produkt aus der Normalkraft $\vec{F}_N$ und dem 
Reibungskoeffizienten $\mu$. Beim Reibungskoeffizienten wird  zwischen 
Haft- und Gleitreibung unterschieden ($\mu_{HR}$ und $\mu_{GR}$).
\[ \boxed{\vec{F}_R = \vec{F}_N \cdot \mu} \]

\subsection{Lösungsvorgehen}
Um ein Hangproblem wie im Bild \ref{fig:hangsystem} zu lösen kann die folgende
Methode angewendet werden.
\begin{enumerate}
	\item Qualitative Skizze erstellen mit sämtlichen Kräften.
	\item Jede Masse inkl. zugehörigen Kräften als eigenes System kennzeichnen.
	\item Systembezogene Gleichungen aufstellen in der Form
		\[ \begin{array}{ll}
			m_1 \cdot \vec{a} & = -\vec{F}_{H_1} -\vec{F}_{R_1} +\vec{F}_S \\
			m_2 \cdot \vec{a} & = +\vec{F}_{H_2} -\vec{F}_{R_2} -\vec{F}_S \\
		\end{array} \]
		wobei nur jene Kräfte summiert werden welche in der Wirkungslinie liegen.
	\item Gleichungssystem lösen durch Addition.
		\[ \vec{a}(m_1 + m_2) = 
			-\vec{F}_{H_1} +\vec{F}_{H_1} -\vec{F}_{R_1} +\vec{F}_{R_2} \]
		\[  \vec{a} = \frac{\sum \vec{F}}{\sum m} \]
	\item Seilkräfte bestimmen mittels der ermittelten Beschleunigung $\vec{a}$.
		\[ \vec{F}_S = \dots \]
\end{enumerate}
		% Schiefe Ebene 
\chapter{Feder}

Die Feder stellt in der Mechanik ein praktisches Bauteil zur Anwendung
des Hooke'schen Gesetzes. Dieses Gesetzt wurde von Robert Hooke 1678 
fomuliert und beschreibt das elastische Verhalten von Festkörpern,
deren elastische Verformung linear zu angewandten Kraft ist. 
In Verbindung mit dem dritten Newton'schen Axiom (Aktion und Reaktion) 
lässt sich auch eine Federkraft beschreiben welche linear zur Stauchung
ist.

\newpage
\section{Definition}

\begin{figure}[h!]
	\centering
	\includegraphics[scale=0.75]{../fig/feder-dehnung.pdf}
	\caption{Veranschaulichung der Federdefinition.}
	\label{fig:feder-dehnung}
\end{figure}

\noindent
Eine Feder kann gedrückt oder gezogen werden, wobei eine Kraft $\vec{F}$ 
entsteht, welche proportional zur Dehnung $\vec{s}$ ist. 
Die Proportionalität wird durch die sogenannte Federkonstante $k$ 
beschrieben. Dies wird auch als Hook'sches Gesetz bezeichnet.

\[ \boxed{\vec{F}_{Feder} = k \cdot \vec{s}} \]

\noindent
Die Richtung der Kraft ist hierbei stets entgegen der Dehnung, d.h. wird die
Feder gestreckt, so zeigt die Kraft zur Feder hin und wird die Feder
gestaucht, so zeigt die Kraft von der Feder weg.

\section{Energie}\label{sec:feder-energie}
Die Energie einer Feder ist definiert als die Federkonstante $k$ multipliziert
mit dem Integral der Dehnung (Weg $\Delta \vec{s}$ bzw. $\vec{s}$).
\begin{figure}[h!]
	\centering
	\includegraphics[scale=0.9]{../fig/feder-energie.pdf}
	\caption{Energie einer Feder als Fläche dargestellt.}
	\label{fig:feder-energie}
\end{figure}

\[ \boxed{E_{Feder} 
	= k \cdot \int_{\vec{s}_a}^{\vec{s}_b} \vec{s} \cdot d\vec{s} 
	= \frac{1}{2} \cdot k \cdot \vec{s}^2
} \] \\

\section{Kombination}
Federn können nicht nur einzeln sondern auch in Kombinationen vorkommen. 
Hierbei werden im speziellen zwei Fälle unterschieden:

\begin{itemize}
	\item Parallele Anordnung von Federn
	\item Serielle Anordnung von Federn
\end{itemize}

\begin{figure}[h!]
	\centering
	\begin{subfigure}[c]{0.45\textwidth}
		\includegraphics[scale=0.75]{../fig/feder-parallel.pdf}
		\caption{Parallele Federanordnung.}
		\label{fig:feder-parallel}
	\end{subfigure}
	\begin{subfigure}[c]{0.45\textwidth}
		\includegraphics[scale=0.75]{../fig/feder-seriell.pdf}
		\caption{Serielle Federanordnung}
		\label{fig:feder-seriell}
	\end{subfigure}
	\caption{Federanordnungen in Vergleich}
	\label{fig:federanordnungen}
\end{figure}

\subsection{Parallele Anordnung}

\noindent
Werden Federn parallel angeordnet, so können die Federkosntanten summiert 
werden.

\[ \boxed{k_{parallel} 
	= \sum_{i=1}^n k_i 
	= k_1 + k_2 + \dots + k_n
} \]

\subsection{Serielle Anordnung}

\noindent
Werden Federn seriell angeordnet, so können die Federkonstanten reziprok 
summiert werden.
\[ \boxed{\begin{array}{l l}\displaystyle
	\frac{1}{k_{seriell}} 
		&= \displaystyle \sum_{i=1}^n \frac{1}{k_i} = 
		\displaystyle \frac{1}{k_1} + \frac{1}{k_2} 
			+ \dots + \frac{1}{k_n} \\
	& \\
	k_{seriell} 
		&= \displaystyle \frac{1}{\left( 
			\displaystyle \sum_{i=1}^n\frac{1}{k_i}\right)} 
		= \displaystyle \frac{1}{\left(
			\frac{1}{k_1} + \frac{1}{k_2} 
			+ \dots + \frac{1}{k_n} \right)}
\end{array}} \]
Analog zur Elektrotechnik hat ein System bestehend aus zwei Federn
in serieller Anordnung spezielle Verhältnisse zwischen Federkonstanten 
und Energie.
\[ \boxed{\begin{array}{r c l  r c l}
\displaystyle
\frac{E_1}{E_{Total}} 
	&=& \displaystyle \frac{k_1 \cdot {k_2}^2}{k_1 + k_2} 
	\qquad \qquad
	& \qquad \displaystyle \frac{E_2}{E_{Total}}
	&=& \displaystyle \frac{{k_1}^2 \cdot k_2}{k_1 + k_2} \\
 & & & & & \\
\displaystyle \frac{E_1}{E_2} 
	&=& \displaystyle \frac{k_2}{k_1} 
	\qquad \qquad
	& \qquad \displaystyle \frac{E_2}{E_1} 
	&=& 
	\displaystyle \frac{k_1}{k_2}
\end{array}} \]
\section{Federkonstante}
Die Federkonstante ist eine material- und geometrieabhängige Grösse welche
berechnet werden kann mittels des sogenannten Elastizitätsmoduls $E$, der
Querschnittsfläche $A$ und der neutralen Länge $l_0$.

\[ \boxed{k = \frac{E \cdot A}{l_0}} \]

       		% Feder
\chapter{Impuls}

\section{Definition}
Der Impuls ist eine Grösse aus der klassischen Mechanik die durch das zweite
Newtonsche Axiom formuliert wurde als das Produkt aus Masse und 
Geschwindigkeit.

\[ \boxed{\vec{p}=m \cdot \vec{v}}  \] 
\[ \sum\vec{F}=\vec{F}_{Res} = m \cdot \frac{d\vec{v}}{dt} =
	\frac{d}{dt}(m \cdot \vec{v}) = \frac{d\vec{p}}{dt}  \]

\noindent
Der Impuls ist somit eine vektorielle Grösse und kann somit auch wie die
Kraft und Geschwindigkeit komponentenweise zusammengetragen werden.

\[\begin{pmatrix} 
	p_{x_1} + p_{x_2} + \dots + p_{x_n} \\
	p_{y_1} + p_{y_2} + \dots + p_{y_n} \\
	p_{z_1} + p_{z_2} + \dots + p_{z_n} 
\end{pmatrix}
=
\begin{pmatrix}
	p_x \\
	p_y \\
	p_z
\end{pmatrix}
= \vec{p}  \]

\section{Kraftstoss}
Der Kraftstoss $\vec{J}$ beschreibt eine Impulsänderung welche sich aus einer 
Kraft $\vec{F}$ und deren Einwirkungsdauer $\Delta t$ ergibt. Der Kraftstoss 
ist definiert als das Integral der angewandten Kraft über die Einwirkungsdauer.

\[ \boxed{\vec{J} = \int_{t_1}^{t_2} \left(\sum \vec{F} \right) dt =
	\int_{t_1}^{t_2}\vec{F}_{Res}(t)\cdot dt = \vec{p}_2 - \vec{p}_1} \]

\subsection{Durchschnittliche Kraft}
Mit der vorgerigen Definition des Kraftstosses $\vec{J}$ lässt sich nun die
durchschnittliche Kraft im Zeitintervall $\Delta t = t_2 - t_1$ bestimmen.

\begin{figure}[h!]
	\centering
	\includegraphics[scale=0.75]{kraftstoss.pdf}
	\caption{Kraftstoss und durchschnittliche Kraft}
	\label{fig:kraftstoss}
\end{figure}

\[ \boxed{\vec{F}_{Av} = \frac{1}{\Delta t} \int_{t_1}^{t_2} \vec{F}(t)dt = 
	\frac{\vec{J}}{\Delta t} = \frac{\vec{p}_2 - \vec{p}_1}{t_2 - t_1}} \]

\section{Impulserhaltung}
In einem abgeschlossenem System\footnote{abgeschlossenes Intertialsystem} 
ist der Gesamtimpuls stets konstant. Der Gesamtimpuls ist die Summe aller 
Impulse im System. Die Impulserhaltung gilt ausnahmslos, d.h. auch bei 
Stössen.

\[ \vec{p} =
\begin{pmatrix}
	p_{x_1} + p_{x_2} + \dots + p_{x_n} \\
	p_{y_1} + p_{y_2} + \dots + p_{y_n} \\
	p_{z_1} + p_{z_2} + \dots + p_{z_n} 
\end{pmatrix} \]

\section{Stoss}
Mit einem Stoss meint man in der Mechanik eine kurze Wechselwirkung zwischen
verschiedenen Körpern. Hierbei gilt grundsätzlich immer die Impulserhaltung
und je nach Stoss auch die Energieerhaltung.

\subsection{Elastischer Stoss}
Ein elastischer Stoss beschreibt ein Zusammentreffen von Körpern wobei die
Energie erhalten bleibt, d.h. es gilt der Impulserhaltungssatz als auch der
Energieerhatlungssatz. Ein bekanntes Beispiel ist das Kugelstosspendel.

\begin{figure}[h!]
	\centering
	\includegraphics[scale=0.4]{kugelstosspendel.pdf}
	\caption{Kugelstosspendel (Newtons Wiege)}
\end{figure}

\[ \boxed{
	\begin{array}{rcl}
		\vec{p}_{vorher} &= &\vec{p}_{nachher} \\
		E_{kin_{vorher}} &= &E_{kin_{nachher}}
	\end{array}
}\]

\noindent
Für einen vollkommen elastischen Stoss zweier Körper lässt sich z.B. mit
den Erhaltungssätzen ermitteln, dass die Geschwindigkeiten nach dem Stoss
in einem festen Verhältnis stehen müssen.
\[ \boxed{\begin{array}{r c l}
	\text{Körper } A,B & & \text{Zeitpunkte } 1,2  \\
	& & \\
	\vec{v}_{A_2} & = & \displaystyle
		\frac{(m_A-m_B)\cdot\vec{v}_{A1}
			+2\cdot m_B\cdot\vec{v}_{B_1}}
			{m_A + m_B} \\
	& & \\
	\vec{v}_{B_2} & = & \displaystyle
		\frac{(m_B-m_A)\cdot\vec{v}_{B_1}
			+2\cdot m_A \cdot \vec{v}_{A_1}}
			{m_A + m_B}
\end{array}} \]

\subsection{Inelastischer Stoss}
Ein inelastischer Stoss beschreibt ein Zusammentreffen von Körpern wobei ein
Teil der kinetischen Energie in Verformungsarbeit übergeht. Somit gilt beim
inelastischen Stoss lediglich die Impulserhaltung. 

\[ \boxed{
	\begin{array}{rcl}
		\vec{p}_{vorher} &= &\vec{p}_{nachher} \\
		E_{kin_{vorher}} &\neq &E_{kin_{nachher}}
	\end{array}
} \]

\noindent
Es gibt beim inelastischen Stoss noch den Spezialfall des vollkommen 
inelastischen Stosses. Dies beschreibt ein solches Zusammentreffen, so dass
der maximal mögliche Anteil der kinetischen Energie in innere Energie
umgewandelt wird. Geschieht solch ein Stoss so kleben die Körper zusammen
und bewegen sich mit der selben Geschwindigkeit fort.

\[ \boxed{
	\begin{array}{rcl}
		\vec{p}_{vorher} &= & \vec{p}_{nachher} \\
		\vec{p_1} + \dots + \vec{p}_n &= &\vec{p}_{nachher} \\
		m_1 \cdot \vec{v_1} + \dots + m_n \cdot \vec{p}_n&= 
			& (m_1 + \dots + m_n)\cdot \vec{v}_{nachher} \\
		 & & \\
		\Rightarrow \vec{v_{cm}} & = & \displaystyle 
			\frac{m_1\cdot\vec{v}_1+\dots+m_n\cdot\vec{v}_n}
				{m_1 + \dots + m_n}
	\end{array}
} \]

\noindent
Ein gutes Beispiel für solch einen vollkommen inelastischen Stoss ist
das Zusammentreffen von Knetmassen. Diese setzen die kinetische Energie
sehr gut in Verformung um.

\section{Raketengleichung}\label{sec:raketengleichung}
Ein System mit zeitlich veränderlicher Masse, kann mit Hilfe von zwei
Gesetzmässigekeiten hin berechnet werden.
\begin{itemize}
	\item Inelastischer Stoss (eher kompliziert)
	\item Gesamtimpuls (eher einfach)
\end{itemize}
Der Gesamtimpuls $\sum \vec{F}_{extern} = \frac{d\vec{P}}{dt}$ wird nun
nicht nur durch die bewegte Masse, sondern auch durch den Massenstrom 
$\dot{m}=\frac{dm}{dt}$ gebildet. 
Mit der Grenzwertbetrachtung nach der Zeit $\frac{d}{dt}$ kann die
Betrachtung vereinfacht werden auf die folgende Form, welche auch
als Raketengleichung bezeichnet wird.
\[  \boxed{
	\sum \vec{F}_{extern} 
		= m \cdot \frac{dv}{dt} 
		- \vec{v}_{relativ} \cdot \frac{dm}{dt}
	} \]
Betrachtet man eine Rakete als System veränderlicher Masse, so können 
den Termen aus der obigen Formel sinnvolle Bezeichnungen zugeordnet 
werden wie Schub- oder Grawitationskraft.
\begin{figure}[h!]
	\centering
	\includegraphics[scale=0.8]{rakete.pdf}
	\caption{Rakete (l) und deren vereinfchtes Modell (r).}
	\label{fig:rakete}
\end{figure}

\[ \boxed{\begin{array}{l l}
\text{Raketengleichung}
	& \sum \vec{F}_{extern} 
	= \underbrace{
		m \cdot \frac{dv}{dt}}_\text{bremsend}
	- \underbrace{
		\vec{v}_{relativ} \cdot \frac{dm}{dt}}_\text{treibend (Schub)} \\
& \\
\text{Gravitation}
	& \vec{F}_G = \displaystyle
		m \cdot \frac{dv}{dt} = m \cdot \vec{a} 
		\quad \rightarrow \vec{a} = g \\
& \\
\text{Schub}
	& \vec{F}_{Schub} = \displaystyle
		\vec{v}_{relativ} \cdot \frac{dm}{dt} \\
& \\
\text{Beschleunigung}
	& \vec{a}(t) = \displaystyle
		\frac{\vec{v}_{relativ}}{m} 
		\cdot \frac{dm}{dt} - g 
	= \displaystyle
		\frac{\vec{v}_{relativ}\cdot \frac{dm}{dt}}{
		m_0 - \frac{dm}{dt} \cdot t} - g \\
 & \\
\text{Geschwindigkeit}
	& \vec{v}(t) = \displaystyle
		\vec{v}_{relativ} 
		\cdot ln \left(\frac{m_0}{m_0-\frac{dm}{dt}\cdot t}\right)
	- g \cdot t 
\end{array} }\]
			    % Impuls
\chapter{Arbeit und Energie}

Energie ist ein zentraler Begriff der Pyhsik und eine wichtige 
Erhaltungsgrösse. Die Arbeit beschreibt im Grunde genommen das selbe wie 
Energie, jedoch wird es sprachlich anders eingesetzt. Als Arbeit 
bezeichent man in der Regel jene Energie, welche mittels eines Weges 
formuliert werden kann und von einem Körper auf einen anderen übertragen 
wird. Die Energie wird oft vereinfacht beschrieben als die Fähigkeit
Arbeit zu verrichten.

\newpage
\section{Arbeit}
Arbeit ist ein Ausdruck der verwendet wird um eine mechanische 
Energieübertrgung zu beschreiben. Diese beschreibt das Produkt aus
Kraft und Weg, wobei diese vektoriell sind.
\[ \boxed{ W = \vec{F} \cdot \vec{s} 
	= ||\vec{F}|| \cdot ||\vec{s}|| \cdot 
	\sphericalangle \left( \vec{F}, \vec{s} \right) } 
	\qquad \left[ J = N \cdot m = \frac{kg \cdot m^2}{s^2} \right] \]

\subsection{Goldene Regel der Mechanik}
Die goldene Regel der Mechanik besagt, dass man Arbeit als konstantes 
Produkt aus Kraft und Weg betrachten kann. Dies ist die Definition des
Energieerhaltungssatzes für den Bereich der Mechanik (z.B. einfache
Machienen wie Flaschenzüge). 
\[ \boxed{W = \vec{F} \cdot \vec{s}} \] 
Dies bedeutet, man kann für die gleiche Arbeit jeweils einen verschieden
langen Weg nehmen. So kann die hierzu nötige Kraft variiert werden. 
Wichtig bei der Anwendung dieser goldenen Regel ist, dass der Weg als
auch die Kraft als vekorielle Grössen verstanden werden. 

\section{Energie}
Die Energie ist eine fundamentale Grösse aller Teilbereiche der Physik.
Diese kann in verschiedenen Formen auftreten. In der Mechanik werden vor
allem die potentielle und kinetische Energie behandelt.

\subsection{Potentielle Energie}
Die potentielle Energie ist definiert als jene Energie, welche sich aus
einem Höhenunterschied (Potentialdifferenz) einer Masse zu einer 
Referenzhöhe ergibt. 
\[ \boxed{ E_{Pot} 
	= m \cdot \vec{g} \cdot (h - h_{Ref})
	= m \cdot \vec{g} \cdot \Delta h 
} \]

\subsection{Kinetische Energie}
Die kinetische Energie ist proportional zur Masse des sich bewegenden Körpers
und quadratisch zu deren Geschwindigkeit. Wichtig ist hierbei auch die 
Betrachtung der Bewegung des Körpers, vor allem zu welcher Referez die 
Geschwindigkeit gilt (Relativgeschwindugkeit).
\[ \boxed{ E_{Kin_{Translation}} = \frac{m \cdot \vec{v}^2}{2} } \]
Die kinetische Energie kann aber auch eine Rotation beschreiben.
\[ \boxed{ E_{Kin_{Rotation}} = \frac{I \cdot \omega^2}{2} } \]

\subsection{Federenergie}
Die Federengergie ist definiert als das Produkt aus Federkonstante $k$ und 
dem Integral der Dehnung bzw. des Weges $s$ 
(siehe Kapitel \ref{sec:feder-energie}).
\[ \boxed{E_{Feder} 
	= k \cdot \int_{\vec{s}_a}^{\vec{s}_b} \vec{s} \cdot d\vec{s} 
	= \frac{k \cdot \vec{s}^2}{2}} \]

\subsection{Reibungsenergie}
Analog zur potentiellen Energie ist die Reibungsenerige durch eine Kraft
$\vec{F}_N \cdot \mu$ und einen Weg $\vec{s}$ definiert. 
$\vec{F}_N$ ist dabei die Kraft, welche senkrecht zur reibenden Unterlage 
wirkt, also $\vec{F}_N \bot \vec{s}$.
\[ \boxed{E_{Reibung} = \vec{F}_N \cdot \mu \cdot \vec{s}} \]
Diese kann auch allgemein formuliert werden mittels des Integrals über
den zurückgelegten Weg.
\[ \boxed{
	E_{Reibung} = \mu \cdot \int \left( 
		\vec{F}_N \cdot \vec{s} \right) ds} 
\]

\section{Leistung}\label{sec:leistung}
Die Leistung $P$ einer Kraft ist die Rate, mit der sie Arbeit zuführt,
also die Arbeit pro Zeit.
\[ \boxed{P = \frac{dW}{dt} } \]
Mit der Überlegung, dass die Arbeit $W$ das Produkt aus Kraft und Weg
ist, kann die Leistung auch als $\vec{F} \cdot \vec{s}$ beschrieben
werden. Mit dieser Formulierung lässt sich die Arbeit auch als Kraft mal
Geschwindigkeit aufstellen.
\[ \boxed{
	P 
		= \frac{dW}{dt} 
		= \frac{\vec{F} \cdot d\vec{s}}{dt}
		= \vec{F} \cdot \frac{d\vec{s}}{dt} 
		= \vec{F} \cdot \vec{v}
} \]
Hierbei gilt es wieder zu beachten, dass dies vektorielle Grössen sind
und der jeweilige Winkel dazwischen zu berücksichtigen ist.

\subsection{Mittlere Leistung}
Mit der Definition aus den Kapitel \ref{sec:leistung} kann nun auch eine
mittlere Leistung formuliert werden mittels der Integration zwischen
zwei Zeitpunkten $t_1$ und $t_2$.
\[ \boxed{
	P_{average} 
		= \frac{\displaystyle\int_{t_1}^{t_2} P\cdot dt}
			{t_2 - t_1}
		= \frac{\Delta W}{\Delta t} 
} \]
		% Energie
\chapter{Schwerpunkt}

\section{Definition}
Der Schwerpunkt eines Systems bezeichnet eine vektorielle Grösse,
welche das Zentrum der Masse auszeichnet. Dieser wird bestimmt
durch den Quotienten aus der Summe aller Massen zu einem Abstand
und der Gesamtmasse.
\[ \boxed{ cm =  
	\begin{pmatrix} 
		x_{cm} \\ 
		\\
		y_{cm} \\
		\\
		z_{cm}
	\end{pmatrix} 
	=
	\begin{pmatrix}
		\frac{m_1 x_1 + m_2 x_2 + \dots + m_n x_n}
			{m_1 + m_2 + \dots + m_n} \\
		\\
		\frac{m_1 y_1 + m_2 y_2 + \dots + m_n y_n}
			{m_1 + m_2 + \dots + m_n} \\
		\\
		\frac{m_1 z_1 + m_2 z_2 + \dots + m_n z_n}
			{m_1 + m_2 + \dots + m_n}
	\end{pmatrix} }
\]
Der Ursprung des Koordinatesnsystems spielt bei der Ermittlung des
Schwerpunktes keine Rolle, denn es ist ein geometrischer Ort des 
betrachteten Objektes. Eine Verscheibung des Koordinatesnsystems
verändert diesen nicht.

\section{Bewegung des Schwerpunktes}

\subsection{Geschwindigkeit}
Wendet man das Newton'sche Bewegungsgesetz auf ein System von
Massen an, so gilt die Bewegung lediglich für den Schwerpunkt,
d.h. alle anderen Massenpnkte welche nicht den Schwerpunkt abbilden,
habe eine andere Bewegung bzw. können eine andere Bewegung haben.
\[ \boxed{ \vec{v}_{cm} = 
	\begin{pmatrix} 
		\frac{m_1 \vec{v_x}_1 
			+ m_2 \vec{v_x}_2 
			+ \dots m_n \vec{v_x}_n}
			{m_1 + m_2 + \dots + m_n} \\
		\\
		\frac{m_1 \vec{v_y}_1 
			+ m_2 \vec{v_y}_2 
			+ \dots m_n \vec{v_y}_n}
			{m_1 + m_2 + \dots + m_n} \\
		\\
		\frac{m_1 \vec{v_z}_1 
			+ m_2 \vec{v_z}_2 
			+ \dots m_n \vec{v_z}_n}
			{m_1 + m_2 + \dots + m_n}
	\end{pmatrix} }
\]
Um ein praktisches Beispiel zu geben kann man sich einen Ball vorstellen.
Der Ball (Schwerpunkt) bewegt sich mit $\vec{v} = m \cdot \vec{a}$.
Die einzelnen Massenpunkte (z.B. Ventil des Balls) muss dieser Bewegung
des Schwerpunktes nicht gleich sein, denn es kann um den Ball herum
rotieren, d.h. eine andere Geschwidigkeit haben und auf einer anderen
Bahn durch den Raum fliegen.

\subsection{Impuls}
Betrachtet man ein System aus Massen, d.h. ein Objekt aus 
Massenpunkten\footnote{Ein Objekt aus Massepunkten ist alles ausser 
einer rein theoretischen Punktmasse ohne Volumen, d.h. ein reales Objekt
wie man es aus dem Alltag kennt.}, so kann
auf dieses System mit Hilfe des Schwerpunktes ein Gesamtimpuls
formuliert werden.
\[ \boxed{
	\left( \sum_{i=1}^n m_i \right) \cdot \vec{v}_{cm} 
		= m_1 \vec{v}_1 + m_2 \vec{v}_2 + \dots m_n \vec{v}_n
		=\vec{p} }
\]

\subsection{Beschleunigung}
Ausgehend von der Geschwindigkeit kann die Beschleunigung des 
Massensystems bestimmt werden. Dies erreicht man durch die 
Ableitung des Gesamtimpulses nach der Zeit $dt$.
\[ \boxed{
	\left( \sum_{i=1}^n m_i \right) \cdot \vec{a}_{cm} 
		= m_1 \vec{a}_1 + \dots m_n \vec{a}_n
		= \frac{d \vec{P}}{dt} 
		= \vec{F}_{Res}
		= \sum_{i=1}^n \vec{F}_{extern} }
\]
Dies beschreibt das Newton'sche Gesetzt welches besagt, dass die 
Summe der anliegenden (externen) Kräfte das Produkt aus Masse und
Beschleunigung bildet. Diese vereinfachte Formulierung bezieht sich
wiederum auf den Schwerpunkt und deren Bewegung.


		% Schwerpunkt
\chapter{Kreisbewegung}
Bei einer Kreisbewegung verläuft die Bahnkurve kreisförmig,
d.h. die bewegte Masse hat eine zusammengesetzte Beschleunigung.
Diese setzt sich aus einem radialen und tangentialen Anteil 
zusammen. Da eine beschleunigte Masse eine Kraft ausübt, wird 
bei einer Kreisbewegung auch von Radial- bzw. Zentripetalkraft
(eine nach innen, zum Zentrum gerichtete Kraft) und der 
Tangentialkraft (eine tangential anliegende Kraft) gesprochen.

\section{Gleichförmige Kreisbewegung}
Bei einer gleichförmigen Kreisbewegung ist die Bahngeschwindigkeit
konstant und hat einen konstanten Abstand (Radius) zum Mittelpunkt.

\begin{figure}[h!]
	\centering
	\includegraphics[scale=0.8]{kreisbewegung.pdf}
	\caption{Gleichförmige Kreisbewegung}
	\label{fig:kreisbewegung}
\end{figure}

\noindent
Die radial wirkende Kraft welche auch als Zentripetalkraft 
bezeichnet wird, kann auf verschiedene Weise berechnet werden.
\[ \boxed{\vec{F}_{rad} = \vec{F}_{ZP} 
	= m \cdot \vec{a}_{rad}
	= m \cdot \frac{\vec{v}^2}{r} 
	= m \cdot \omega^2 \cdot r 
	= m \cdot \frac{4 \cdot \pi^2 \cdot r}{T^2} } \]
Die Beziehungen zwischen den Bahngrössen lassen sich dabei 
grundsätzlich mit vier Gleichungen beschreiben:
\[ \boxed{ \begin{array}{l r l}
	\text{Bogenlänge $s$} & 
		s & = r \cdot \theta \\
	& & \\
	\text{Geschwindigkeit tangential $\vec{v}$} &
		\vec{v} & = r \cdot \omega
		= r \cdot \frac{d\omega}{dt} \\
	& & \\
	\text{Beschleunigung tangential $\vec{a}_{tan}$} &
		\vec{a}_{tan} & = r \cdot \alpha 
		= r \cdot \frac{d^2\theta}{dt} \\
	& & \\
	\text{Beschleunigung radial $\vec{a}_{rad}$} &
		\vec{a}_{rad} & = \frac{v^2}{r} = \omega^2 \cdot r
\end{array} }\]

\noindent
Wichtig ist der Zusammenhang zwischen den verschiedenen Beschleunigungen
eines rotierenden Körpers zu beachten.
\[ \boxed{\begin{array}{r l}
	\vec{a}_{rad} & = \frac{\vec{v}^2}{r} 
		= \omega^2 \cdot r 
		= \dot{\theta}^2 \cdot r \\
	& \\
	\vec{a}_{tan} & = \alpha \cdot r
		= \ddot{\theta} \cdot r \\
	& \\
	\vec{a}_{res} & = \sqrt{\vec{a}_{rad}^2 + \vec{a}_{tan}^2} 
\end{array} }\]

\section{Winkelbetrachtung}\label{sec:winkelbetrachtung}
Bei Kreisbewegungen ist es von Vorteil, wenn die Bewegung nicht als
Weg pro Zeit ($\frac{m}{s}$) sondern als Winkel pro Zeit 
($\frac{rad}{s}$) betrachtet wird. Eine Kreisbewegung kann somit 
unabhängig vom Radius beschrieben werden. In diesem Fall spricht
man dann von der Winkelgeschwindigkeit $\omega$ und 
Winkelbeschleunigung $\alpha$. Mit dieser Betrachtungsweise kann
eine Kreisbewegung analog zur Translation beschrieben werden, denn
auch hier gilt der Zusammenhang von Weg, Geschwindigkeit und 
Beschleunigung wie in der Translation.

\[ \boxed{ \begin{array}{l l}
	\text{Rotation} \qquad &
		\theta
		\xrightarrow{\frac{d}{dt}} \omega 
		\xrightarrow{\frac{d}{dt}} \alpha \\
	& \\
	\text{Translation} \qquad &
		s
		\xrightarrow{\frac{d}{dt}} \vec{v} 
		\xrightarrow{\frac{d}{dt}} \vec{a}
\end{array} }\]

\noindent
Die einzelnen Grössen $\omega$ und $\alpha$ können mit den folgenden 
Formeln beschrieben werden.
\[ \boxed{\begin{array}{l r l}
	\text{Winkelgeschwindigkeit} &
		\omega & = \omega_0 + \alpha \cdot t 
			= \frac{d\theta}{dt} 
			= \dot{\theta} \\
	& & \\
	\text{Winkelbeschleunigung} &
		\alpha & = \frac{d^2\theta}{dt^2} = \ddot{\theta}
\end{array} }\]

\section{Trägheitsmoment}
Das Trägheitsmoment beschreibt die Massenverteilung eines Körpers
bezogen auf eine Drehachse. Dieses muss bei jeder Form von Rotation
berücksichtigt werden und ist grundsätzlich unabhängig von der 
Bewegung, nicht aber von der Drehachse.

Das Trägheitsmoment wird beschrieben durch die Summe der 
Produkte aus den Massepunkten und deren Abstand im Quadrat zur 
Drehachse des Körpers.
\[ \boxed{
	I = m_1 \cdot r_1^2
		+ m_2 \cdot r_2^2 
		+ \dots 
		+ m_n \cdot r_n^2
		= \sum m_i \cdot r_i^2
		= \int r^2 \cdot dm
} \]

\noindent
Trägheitsmomente verschiedener Körper können zu einem Trägheitsmoment
summiert werden, wenn diese auf der gleichen Drehachse rotieren.
\[ \boxed{
	I_{Total} = I_1 + I_2 + \dots + I_n
} \]

\subsection{Trägheitsmomente regelmässiger Körper}


\section{Rotationsenergie}
Die Rotationsenergie ist im Grunde genommen die kinetische Enerige
eines starren Körpers der um eine feste Achse rotiert.
\[ \boxed{\begin{array}{r l}
	E_{kin} & = \frac{1}{2} \cdot m \cdot \vec{v}^2 \\
	& \\
	\Rightarrow E_{rot} & = \frac{1}{2} \cdot m_1 \cdot \vec{v}_1^2
			+ \frac{1}{2} \cdot m_2 \cdot \vec{v}_2^2
			+ \dots 
			+ \frac{1}{2} \cdot m_n \cdot \vec{v}_n^2 \\
	& \\
		& = \frac{1}{2} \cdot \left(m_1 \cdot r_1^2
			+ m_2 \cdot r_2^2 
			+ \dots
			+ m_n \cdot r_n^2\right) \cdot \omega^2 \\
	& \\
		& = \frac{1}{2} \cdot I \cdot \omega^2 
\end{array} } \]

\newpage
\section{Rotation vs. Translation}
Wie im Kapitel \ref{sec:winkelbetrachtung} beschrieben, kann die 
Rotation analog zur Translation betrachtet werden.

\[ \boxed{\begin{array}{l l}
\textbf{Rotation} & \textbf{Translation} \\
& \\
\text{Winkelbeschleunigung $\left[\frac{rad}{s^2}\right]$}
	& \text{Beschleunigung $\left[\frac{m}{s^2}\right]$} \\
		\qquad \alpha = \frac{d^2\theta}{dt^2} = \ddot{\theta}
			& \qquad \vec{a} = \frac{d^2s}{dt^2} \\
& \\
\text{Winkelgeschwindigkeit $\left[\frac{rad}{s}\right]$} 
	& \text{Geschwindigkeit $\left[\frac{m}{s}\right]$} \\
		\qquad \omega = \frac{d\theta}{dt} = \dot{\theta} 
			& \qquad \vec{v} = \frac{ds}{dt} \\
& \\
\text{Winkel $\left[rad\right]$} 
	& \text{Weg $\left[ m \right]$} \\
		\qquad \theta = \omega_0 \cdot t + \frac{1}{2} \cdot \alpha \cdot t^2
			& \qquad s = \vec{v}_0 \cdot t + \frac{1}{2} \cdot \vec{a} \cdot t^2 \\
		& \\
		\qquad \theta = \frac{1}{2} \cdot (\omega + \omega_0) \cdot t
			& \qquad s = \frac{1}{2} \cdot (\vec{v} + \vec{v}_0) \cdot t \\
& \\
\text{Trägheitsmoment $\left[kg \cdot m^2\right]$} 
	& \text{Masse $\left[kg\right]$} \\
		\qquad I = \sum r_i^2 \cdot m_i = \int r^2 \cdot dm
			& \qquad m \\
& \\
\text{Drehmoment $\left[N \cdot m\right]$}
	& \text{Kraft $\left[N\right]$} \\
		\qquad M = I \cdot \alpha
			& \qquad \vec{F} = m \cdot \vec{a} \\
& \\
\text{Drehimpuls $\left[\frac{kg \cdot m^2}{s}\right]$} 
	& \text{Impuls $\left[N \cdot s\right]$} \\
		\qquad L = I \cdot \omega 
			= r \times \vec{p}
			& \qquad \vec{p} = m \cdot \vec{v} \\
& \\
\text{Rotationsenergie $\left[J\right]$} 
	& \text{Kinetische Energie $\left[J\right]$} \\
		\qquad E_{Rot} = \frac{1}{2} \cdot I \cdot \omega^2
			& \qquad E_{Kin} = \frac{1}{2} \cdot m \cdot \vec{v}^2 \\
& \\
\text{Leistung $\left[W\right]$}
	& \text{Leistung $\left[W\right]$} \\
		\qquad P = \frac{dE}{dt} = M \cdot \omega 
			& \qquad P = \vec{F} \cdot \vec{v}
\end{array} } \]


		% Kreisbewegung
%% coding:utf-8

%----------------------------------------
%FOSAPHY, a LaTeX-Code for a summary of basic physics
%Copyright (C) 2013, Daniel Winz, Ervin Mazlagic

%This program is free software; you can redistribute it and/or
%modify it under the terms of the GNU General Public License
%as published by the Free Software Foundation; either version 2
%of the License, or (at your option) any later version.

%This program is distributed in the hope that it will be useful,
%but WITHOUT ANY WARRANTY; without even the implied warranty of
%MERCHANTABILITY or FITNESS FOR A PARTICULAR PURPOSE.  See the
%GNU General Public License for more details.
%----------------------------------------

\chapter{Rotation}
\section{Zentripetalkraft}
\[ F_z = \frac{m \cdot {v_b}^2}{r} = m \cdot \omega^2 \cdot r \]

\section{Rotationsbewegung}

\subsection{Winkel $\theta$}
\[ \theta =  \]

\subsection{Winkelgeschwindigkeit $\omega$}
\[ \omega = \frac{v_{tan}}{r} \]

\subsection{Winkelbeschleunigung $\alpha$}
\[ \alpha =  \]

\subsection{Drehmoment $M$}
\[ M = I \cdot \alpha = \vec{r} \times \vec{F} = r \cdot F \cdot \sin(\phi) 
= r \cdot F_{tan} \]

\subsection{Trägheitsmoment $I$}
Das Trägheitsmoment ist von der Form des Körpers abhängig. 
\[ I = \sum (m_i \cdot r^2) = \int(r^2)dm\]
Für einige einfache Körper existieren Formeln für die Berechnung des 
Trägheitsmoments. 
\begin{table}[h!]
\begin{tabular}{p{0.5\textwidth}l}
\rowcolor{lgray}
Körper 
& Trägheitsmoment \\

\rowcolor{white}
Stab mit Drehachse in der Mitte, senkrecht zur Symmetrieachse 
& $\dfrac{1}{12} \cdot m \cdot L^2$ \\

\rowcolor{lgray}
Stab mit Drehachse am Ende, senkrecht zur Symmetrieachse 
& $\dfrac{1}{3} \cdot m \cdot L^2$ \\

\rowcolor{white}
Platte mit Drehachse durch Zentrum, senkrecht zur Oberfläche
& $\dfrac{1}{12} \cdot m \cdot (a^2 + b^2)$ \\

\rowcolor{lgray}
Platte mit Drehachse entlang der Kante b
& $\dfrac{1}{3} \cdot m \cdot a^2$ \\

\rowcolor{white}
Dickwandiger Zylinder
& $\dfrac{1}{2} \cdot m \cdot ({r_1}^2 + {r_2}^2)$ \\

\rowcolor{lgray}
Vollzylinder
& $\dfrac{1}{2} \cdot m \cdot r^2$ \\

\rowcolor{white}
Dünnwandiger Zylinder
& $\dfrac{1}{} \cdot m \cdot r^2$ \\

\rowcolor{lgray}
Vollkugel
& $\dfrac{2}{5} \cdot m \cdot r^2$ \\

\rowcolor{white}
Hohlkugel
& $\dfrac{2}{3} \cdot m \cdot r^2$ \\
\end{tabular}
\end{table}

\subsection{Drehimpuls $L$}
\[ L = I \cdot \omega \]

\subsection{Rotationsenergie}
\[ E_{rot} = \frac{1}{2} \cdot I \cdot \omega^2 \]

\subsection{Leistung}
\[ P_{rot} = M \cdot \omega \]
    		% Rotation
% coding:utf-8

%----------------------------------------
%FOSAPHY, a LaTeX-Code for a summary of basic physics
%Copyright (C) 2013, Daniel Winz, Ervin Mazlagic

%This program is free software; you can redistribute it and/or
%modify it under the terms of the GNU General Public License
%as published by the Free Software Foundation; either version 2
%of the License, or (at your option) any later version.

%This program is distributed in the hope that it will be useful,
%but WITHOUT ANY WARRANTY; without even the implied warranty of
%MERCHANTABILITY or FITNESS FOR A PARTICULAR PURPOSE.  See the
%GNU General Public License for more details.
%----------------------------------------

\chapter{Reibung}

Die Reibung bezeichnet in der Pyhsik eine Eigenschaft, welche stets einer
Bewegung entgegenwirkt. In der Mechanik kann sie als Kraft verstanden
werden, welche zwischen sich berührenden Körpern auftritt. Hiebei gibt es
einige Arten der Reibung, die aufgrund ihres speziellen Charakters, 
unterschieden werden. Die wohl wichtigsten sind die Haft- und 
Gleitreibung (weiter gibt es z.B. noch die Roll-, Wälz-, Bohrreibung).

\newpage
\section{Reibungskraft}
Die Reibungskraft $\vec{F}_R$ ist eine Kraft welche zwischen sich
berührenden Körpern auftritt. Sie ist definiert als das Produkt aus
der Kraft $\vec{F}_N$ welche normal zwischen den Körpern anliegt
und dem Reibungskoeffizienten $\mu$ welcher die 
Oberflächenbeschaffenheit beschreibt. 

\begin{figure}[h!]
	\centering
	\includegraphics[scale=0.8]{reibung.pdf}
	\caption{Reibungskraft für eine Masse die auf einer Unterlage 
		liegt mit $\sum \vec{F}_{extern}$ positiv (l)
		und negativ (r).}
\end{figure}

\noindent
Die Richtung der Reibungskraft ist dabei stets entgegen der 
anliegenden äusseren Nettokraft 
$\sum \vec{F}_{extern}$.

\[ \boxed{
	\vec{F}_R = \vec{F}_N \cdot \mu
		\qquad ,\vec{F}_R \bot \vec{F}_N
}\]

\section{Bewegungsbedingung}
\noindent
Ein Körper der eine Reibung kennt, bringt eine Kraft $\vec{F}_R$ 
auf entgegen der anliegenden Nettokraft $\sum \vec{F}_{extern}$. 
Das bedeutet, dass ein solcher Körper in Ruhe bleibt, solange die 
anliegende Nettokraft $\sum \vec{F}_{extern} \leq \vec{F}_R$ ist. 
Somit ergibt sich die Bewegungsbedingung zu

\[ \boxed{\begin{array}{l r l}
	\text{Körper bleibt in Ruhe falls } & 
		\sum \vec{F}_{extern} & \leq \vec{F}_R \\
	& & \\
	\text{Körper bewegt sich falls } &
		\sum \vec{F}_{extern} & > \vec{F}_R 
\end{array}} \]

\section{Reibungskoeffizient}
Der Reibungskoeffizient $\mu$ wird typischerweise für zwei Fälle 
angegeben, welche voneinander zu unterscheiden sind.
\begin{itemize}
	\item Haften $\Rightarrow$ Haftreibungskoeffizient $\mu_{HR}$
	\item Gleiten $\Rightarrow$ Gleitreibungskoeffizient $\mu_{GR}$
\end{itemize}

\noindent
Für den Reibungskoeffizienten gilt stets, dass der Koeffizient für die
Haftung grösser oder zumindest gleich ist wie für das Gleiten 
(bei den selben Bedingungen).

\[ \boxed{
	\mu_{HR} \geq \mu_{GR}
}\]

\begin{footnotesize}
\begin{longtable}{llll}
  \rowcolor{white} \textbf{Material 1} & \textbf{Material 2} 
  & \textbf{Haftreibung $\mu_{HR}$} & \textbf{Gleitreibung $\mu_{GR}$}\\
  \rowcolor{lgray}  Stahl     & Stahl             & 0.74 & 0.57\\
  \rowcolor{white}  Aluminium & Stahl             & 0.61 & 0.47\\
  \rowcolor{lgray}  Kupfer    & Stahl             & 0.53 & 0.36\\
  \rowcolor{white}  Messing   & Stahl             & 0.51 & 0.44\\
  \rowcolor{lgray}  Zink      & Grauguss          & 0.85 & 0.21\\
  \rowcolor{white}  Kupfer    & Grauguss          & 1.05 & 0.29\\
  \rowcolor{lgray}  Glas      & Glas              & 0.94 & 0.40\\
  \rowcolor{white}  Kupfer    & Glas              & 0.68 & 0.53\\
  \rowcolor{lgray}  Teflon    & Teflon            & 0.04 & 0.04\\
  \rowcolor{white}  Teflon    & Stahl             & 0.04 & 0.04\\
  \rowcolor{lgray}  Gummi     & Beton (trocken)   & 1.0  & 0.80\\
  \rowcolor{white}  Gummi     & Beton (nass)      & 0.30 & 0.25
\end{longtable}
\end{footnotesize}

     		% Reibung
% coding:utf-8

%----------------------------------------
%FOSAPHY, a LaTeX-Code for a summary of basic physics
%Copyright (C) 2013, Daniel Winz, Ervin Mazlagic

%This program is free software; you can redistribute it and/or
%modify it under the terms of the GNU General Public License
%as published by the Free Software Foundation; either version 2
%of the License, or (at your option) any later version.

%This program is distributed in the hope that it will be useful,
%but WITHOUT ANY WARRANTY; without even the implied warranty of
%MERCHANTABILITY or FITNESS FOR A PARTICULAR PURPOSE.  See the
%GNU General Public License for more details.
%----------------------------------------

\chapter{Kurvenfahrt}

\newpage

\section{Einfache Kurvenfahrt}
Eine einfache Kurvenfahrt ist eine idealiserte Form der Kreisbewegung,
bei welcher der Radius $r$ normal zur Gravitation $\vec{g}$ liegt und
die Kurve flach ist d.h. die Höhe zu einem beliegiben $r$ ist stets die
selbe, somit ist $h=$konstant. 
Ein ideales Beispiel ist ein Auto, welches eine Kurve fährt.

\begin{figure}[h!]
	\centering
	\includegraphics[scale=0.8]{kurve3.pdf}
	\caption{Einfache Kurvenfahrt aus der Vogelperspektive.}
	\label{fig:kurvenfahrt}
\end{figure}

\noindent
Aus der Kreibewegung geht hervor, dass es eine nach innen gerichtete
Kraft $\vec{F}_Z$ mit der entsprechenden Beschleunigung $\vec{a}_{rad}$
gibt. Die Grafik \ref{fig:kurvenfahrt} verdeutlicht, dass im Beispiel 
eines Autos welches durch eine Kurve fährt, diese nach innen gerichtete 
Kraft ebenfalls vorhanden ist. Da bei einem Auto die Reifen eine Reibung 
stellen, muss die Richtung dieser Kraft zwar in der selben Wirkungslinie
sein wie $\vec{F}_Z$, nicht aber die selbe Richtung besitzen.  

\newpage
\section{Steilkurvenfahrt}
Bei einer Steilkurve ist die Höhe des Innen- und Aussenradius nicht
gleich. Eine Betrachtung zu einem bestimmten Punkt in der Kurve stellt
somit eine Kombination aus Hang- und Kreisbewegung dar. 

\begin{figure}[h!]
	\centering
	\includegraphics[scale=0.8]{kurve1.pdf}
	\caption{Steilkurve mit 
		$\vec{v}_{klein}$ (l) und $\vec{v}_{gross}$ (r)}
	\label{fig:steilkurve1}
\end{figure}

\noindent
Hierbei gilt es zu beachten, dass die Normalkraft $\vec{F}_N$ nicht 
eine Komponente der Gewichtskraft $\vec{F}_G$ ist, d.h. 
$\vec{F}_N \neq \vec{F}_G \cdot cos(\alpha)$. 

Eine solche Steilkurvenfahrt wird grundsätzlich komponentenweise
betrachtet. Hierzu werden die Kräfte $\vec{F}_N$ und $\vec{F}_R$ in
eine $x$- und $y$-Komponente zerlegt.

\begin{figure}[h!]
	\centering
	\includegraphics[scale=0.8]{kurve2.pdf}
	\caption{Steilkurve mit zerlegten Kräften in $x,y$}
	\label{fig:steilkurve2}
\end{figure}

\noindent
Sind die Kräfte zerlegt, so können die Bewegungsgleichungen für die 
jeweilige Komponente $x,y$ formuliert werden. Die Vorzeichen sind 
wiederum nach der gewählten Bewegungsrichtung zu setzen. 

\subsection{Lösungsvorgehen}
Um ein Steilkurvenproblem wie im Bild \ref{fig:steilkurve2} dargestellt
zu lösen, kann die folgende Methode angewandt werden.

\begin{enumerate}
	\item Qualitative Skizze erstellen mit allen relevanten Kräften
		des Systems (z.B. $\vec{F}_G$, $\vec{F}_N$, $\vec{F}_R$, 
		$\vec{F}_Z$).
	\item Die winkelabhängigen Kräfte (z.B. $\vec{F}_N, \vec{F}_R$) 
		in $x,y$-Komponenten zerlegen.
	\item Bewegungsrichtung definieren (welche Kraft ist positiv, 
		welche negativ).
	\item Bewegungsgleichung aufstellen für $x,y$
\[\begin{array}{l r c l}
	x: &
		+ \vec{F}_{N_x} - \vec{F}_{R_x} &=& + \vec{F}_Z \\
	& & & \\
	y: &
		+ \vec{F}_{N_y} + \vec{F}_{R_y} &=& + \vec{F}_G
\end{array} \]

	\item Die Kraft, welche zur Kurve normal ist ermitteln (falls
		$\vec{a}_y=0$, dann kann $\vec{F}_N$ durch die
		Bewegungsgleichung von $y$ ermittelt werden).
	\item Restliche Unbekannten ermitteln.
\end{enumerate}
		%Kurvenfahrt
% coding:utf-8

%----------------------------------------
%FOSAPHY, a LaTeX-Code for a summary of basic physics
%Copyright (C) 2013, Daniel Winz, Ervin Mazlagic

%This program is free software; you can redistribute it and/or
%modify it under the terms of the GNU General Public License
%as published by the Free Software Foundation; either version 2
%of the License, or (at your option) any later version.

%This program is distributed in the hope that it will be useful,
%but WITHOUT ANY WARRANTY; without even the implied warranty of
%MERCHANTABILITY or FITNESS FOR A PARTICULAR PURPOSE.  See the
%GNU General Public License for more details.
%----------------------------------------

\chapter{Schwingung}
\section{Einfache harmonische Schwingung}
Rücktreibende Kraft: 
\[ \boxed{F = - k \cdot x} \]
\begin{tabular}{ll}
$F:$ & Rücktreibende Kraft \\
$k:$ & Federkonstante \\
$x:$ & Auslenkung
\end{tabular}
\[ \boxed{F = m \cdot a(t) = m \cdot \frac{d^2}{d t^2}x(t) = m \cdot \ddot{x}} \]
\[ \boxed{m \cdot \ddot{x} + k \cdot x = 0} \]
\[ \boxed{x(t) = A \cdot \cos(\omega \cdot t + \phi)} \]
\[ \boxed{\omega = 2 \cdot \pi \cdot f = \sqrt{\frac{k}{m}}} \]
\[ \boxed{T = 2 \cdot \pi \sqrt{\frac{m}{k}}} \]
\[ \boxed{\dot{x} = - A \cdot \omega \cdot \sin(\omega \cdot t + \phi)} \]
\[ \boxed{\ddot{x} = - A \cdot \omega^2 \cdot \cos(\omega \cdot t + \phi)} \]
\[ \boxed{m \cdot (-A \cdot \omega^2 \cdot \cos(\omega \cdot t + \phi)) 
+ k \cdot A \cdot \cos(\omega \cdot t + \phi) \stackrel{!}{=} 0} \]
\[ \boxed{A \cdot \cos(\omega \cdot t + \phi) \cdot (-m \cdot  \omega^2 + k) 
\stackrel{!}{=} 0} \]
\[ \boxed{-m \cdot  \omega^2 + k \stackrel{!}{=} 0} \]
\[ \boxed{\omega^2 = \frac{k}{m}} \]
\[ \boxed{\omega = \sqrt{\frac{k}{m}}} \]

\subsection{Auslenkung und Geschwindigkeit als Anfangsbedingungen}
\[ \boxed{x_0 = x(0) = A \cdot \cos(\phi)} \]
\[ \boxed{v_0 = \dot{x}(0) = - \omega \cdot A \cdot \sin(\phi)} \]
\[ \boxed{\phi = \tan^{-1}\left(-\frac{v_0}{\omega \cdot x_0}\right)} \]
\textbf{Achtung!} Lösung kann im falschen Quadranten liegen. 
Evtl. mit $pi$ addieren. 
\[ \boxed{A = \sqrt{{x_0}^2 + \left(\frac{v_0}{\omega}\right)^2}} \]

\subsection{Geschwindigkeit und Beschleunigung aus Amplitude und Position}
\[ \boxed{v(t) = \sqrt{\frac{k}{m}} \cdot \sqrt{A^2 - x^2(t)}} \]
\[ \boxed{a(t) = - \frac{k}{m} \cdot x(t)} \]

\subsection{Maximale Geschwindigkeit und Beschleunigung}
\[ \boxed{v_{max} = \omega \cdot A} \]
\[ \boxed{v_{min} = -\omega \cdot A} \]
\[ \boxed{a_{max} = -\omega^2 \cdot A} \]

\subsection{Vertikales Federpendel}
\[ \boxed{\omega = \sqrt{\frac{k}{m}}} \]
\[ \boxed{x(t) = - \frac{m \cdot g}{k} + A \cdot \cos(\omega \cdot t + \phi)} \]

\subsection{Energie}
\[ \boxed{E_{pot} = \frac{1}{2} \cdot k \cdot x(t)^2} \]
\[ \boxed{E_{kin} = \frac{1}{2} \cdot m \cdot v(t)^2} \]
\[ \boxed{E_{tot} = \frac{1}{2} \cdot m \cdot v^2 + \frac{1}{2} \cdot k \cdot x^2 
= \frac{1}{2} \cdot k \cdot A^2 = \frac{1}{2} \cdot m \cdot \omega^2 \cdot A^2 
= \frac{1}{2} \cdot m \cdot (v_{max})^2} \]

\section{Rotationsschwingung}
\[ \boxed{\omega = \sqrt{\frac{\kappa}{I}}} \]
\[ \boxed{\Theta(t) = \Theta_0 \cdot \cos(\omega \cdot t + \phi)} \]
\[ \boxed{T = 2 \cdot \pi \cdot \sqrt{\frac{I}{\kappa}}} \]

\subsection{Stab am Ende aufgehängt}
\[ \boxed{I = \frac{1}{12} \cdot m \cdot L^2} \]
\[ \boxed{M = -F \cdot r \cdot \sin(\angle) = -k \cdot \Delta x \cdot \frac{L}{2}} \]
\[ \boxed{M = -k \left(\frac{L}{2}\right)^2 \cdot \Theta} \]
Für kleine Winkel: 
\[ \boxed{\kappa = k \cdot \left(\frac{L}{2}\right)^2} \]
\[ \boxed{I = \frac{1}{12} \cdot m \cdot L^2} \]
\[ \boxed{\omega = \sqrt{\frac{\kappa}{I}} 
= \sqrt{\frac{k \cdot \left(\frac{L}{2}\right)^2}{\frac{1}{12}\cdot m \cdot L^2}} 
= \sqrt{\frac{3 \cdot k}{m}}} \]

\section{Physikalisches Pendel}
\[ \boxed{M = -d \cdot m \cdot g \cdot \sin(\Theta) 
\cong - \underbrace{d \cdot m \cdot g}_{\kappa} \cdot \Theta} \]
\[ \boxed{\omega = \sqrt{\frac{\kappa}{I}} 
= \sqrt{\frac{m \cdot g \cdot d}{I_z}}} \]
\[ \boxed{T = 2 \cdot \pi \cdot \sqrt{\frac{I}{m \cdot g \cdot d}}} \]
speziell: Fadenpendel
\[ \boxed{I = m \cdot L^2} \]
\[ \boxed{\kappa = m \cdot g \cdot L} \]
\[ \boxed{2 \cdot \pi \cdot f = \sqrt{\frac{m \cdot g \cdot L}{m \cdot L^2}}} \]
\[ \boxed{f = \frac{1}{2 \cdot \pi} \cdot \sqrt{\frac{g}{L}}} \]
\[ \boxed{\omega = \sqrt{\frac{g}{L}}} \]
\[ \boxed{T = 2 \cdot \pi \cdot \sqrt{\frac{L}{g}}} \]
Fadenkraft: 
\[ \boxed{F_s = m \cdot g \cdot \cos(\theta) + m \cdot \frac{v^2}{L}} \]

\section{Pendelnde Flüssigkeitssäule}
\[ \boxed{\omega = \sqrt{\frac{2 \cdot g}{L}}} \]

\section{Gedämpfte Schwingung}
Fall 1: $\beta > \omega$ "'Kriechfall"'
\[ \boxed{x(t) \sim e^{(-\beta \pm \delta)t}} \]
Fall 2: $\omega = \beta$ "'kritische Dämpfung"'
\[ \boxed{\beta = \frac{b}{2 m} \quad \Leftrightarrow \quad 
b = b_{krit} = \sqrt{4 \cdot k \cdot m}} \]
Fall 3: $\omega > \beta$ "'Gedämpfte Schwingung"'
\[ \boxed{x(t) = A \cdot e^{-\beta t} \cdot \cos(\omega_d t)} \]
\[ \boxed{\omega_d = \sqrt{\omega^2 - \beta^2} 
= \sqrt{\frac{k}{m} - \left({\frac{b}{2 \cdot m}}\right)^2}} \]
Zerfallszeit: 
\[ \boxed{\tau = \frac{1}{\beta}} \]
\[ \boxed{A(t) = A \cdot e^{-\beta t} = A \cdot e^{-\frac{t}{\tau}}} \]
Bei der Zeit $t = \tau$
\[ \boxed{A(\tau) = \frac{A}{e} = A \cdot e^{-1} \approx 0.37 A} \]
Abklingkonstante: 
\[ \boxed{\beta 
= \frac{1}{t_2 - t_1} \cdot \ln\left(\frac{x(t_1)}{x(t_2)}\right)} \]
Schwingungsenergie: 
\[ \boxed{E(t) = \frac{m}{2} \cdot \dot{x}^2(t) + \frac{k}{2} \cdot x^2(t)} \]
\[ \boxed{\frac{d E(t)}{dt} 
= \frac{m}{2} \cdot 2 \cdot \dot{x}^2(t) \cdot \ddot{x}^2(t) 
+ \frac{k}{2} \cdot 2 \cdot x^2(t) \cdot \dot{x}^2(t) 
= \underbrace{(m \ddot{x}(t) + k x(t))}_{-b \dot{x}(t)} \cdot \dot{x}(t)} \]
\[ \boxed{\frac{d E(t)}{dt} = -b \dot{x}^2 - b v^2(t) \leq 0} \]
\[ \boxed{A(t) = A \cdot e^{-\frac{t}{\tau}}} \]
\[ \boxed{E(t) = E_0 \cdot e^{-\frac{2 \cdot t}{\tau}}} \]
Güte: 
\[ \boxed{Q = \frac{\pi}{\beta \cdot T_d} = \pi \cdot \frac{\tau}{T_d} 
= \frac{\omega_d \cdot \tau}{2}} \]

\section{Erzwungene Schwingung}
\[ \boxed{x(t) = A(\Omega) \cdot \cos(\Omega t - \varphi(\Omega))} \]
\[ \boxed{A(\Omega) = \frac{F_0}{m \cdot 
\sqrt{(\omega^2 - \Omega^2)^2 + (2 \cdot \Omega \cdot \beta)^2}} 
\approx \frac{H}{\sqrt{\left(1 - \left(\frac{\Omega}{\omega}\right)^2\right)^2 
+ \left(\frac{\Omega}{Q \cdot \omega}\right)^2}}} \]  
\[ \boxed{\varphi(\Omega) = \arctan\left(\frac{2 \cdot \Omega \cdot \beta}
{{\omega_0}^2 - \Omega^2}\right) 
= \arctan{\left(\frac{\frac{b}{\sqrt{k \cdot m}} 
\left(\frac{\Omega}{\omega}\right)}
{1 - \left(\frac{\Omega}{\omega}\right)^2}\right)}} \]
\[ \boxed{H = \frac{F}{k}} \]
\[ \boxed{\omega = \sqrt{\frac{k}{m}}} \]
\[ \boxed{\beta = \frac{b}{2 m} = \frac{1}{\tau}} \]
\[ \boxed{Q = \frac{\omega_d}{2\beta} = \omega_d \frac{m}{b} 
= \pi \frac{\tau}{T}} \]
schwache Dämpfung: 
\[ \boxed{b << b_{krit} = \sqrt{4 k \cdot m} \quad Q >> 1} \]

\subsection{Resonanzfrequenz}
\[ \boxed{\frac{d A(\Omega)}{d\Omega} = 0} \]
\[ \boxed{\Omega_R = \sqrt{\omega^2 - 2\beta^2} \neq \omega_d 
= \sqrt{\omega^2 - \beta^2} \neq \omega} \]
\[ \boxed{A(\Omega_R) = \frac{\omega^2 H}{2 \beta \sqrt{\omega^2 - \beta^2}} 
= \frac{k H}{m \sqrt{(\omega^2 - {\Omega_R}^2)^2 + (2\beta\Omega_R)^2}} 
\approx \frac{Q H}{\sqrt{1 - \frac{1}{2 Q^2}}} \approx Q \cdot H} \]
\[ \boxed{\varphi_R = \varphi(\Omega_R) 
= \arctan\left(\frac{2 \beta \omega_R}{\omega^2 - {\Omega_R}^2}\right) 
\approx \arctan\left(\sqrt{4 Q^2 - 2}\right) \approx \frac{\pi}{2}} \]
Für $Q>5$
\[ \boxed{\Omega_R \approx \omega \left(1 - \frac{1}{4Q^2}\right)} \]
\[ \boxed{A_R = A(\Omega_R) \approx \frac{Q H}{\sqrt{1 - \frac{1}{2Q^2}}}} \]
\[ \boxed{\Delta\Omega \approx \frac{\omega}{Q}} \]

  		% Schwingung
% coding:utf-8

%----------------------------------------
%FOSAPHY, a LaTeX-Code for a summary of basic physics
%Copyright (C) 2013, Daniel Winz, Ervin Mazlagic

%This program is free software; you can redistribute it and/or
%modify it under the terms of the GNU General Public License
%as published by the Free Software Foundation; either version 2
%of the License, or (at your option) any later version.

%This program is distributed in the hope that it will be useful,
%but WITHOUT ANY WARRANTY; without even the implied warranty of
%MERCHANTABILITY or FITNESS FOR A PARTICULAR PURPOSE.  See the
%GNU General Public License for more details.
%----------------------------------------

\chapter{Wellen}
\section{Allgemein}
\[ \boxed{f(x,t) = f^*(x \pm v \cdot t} \]

\section{Seilwelle}
\begin{tabular}{@{}ll}
$FS$:   & Seilkraft in $kg \frac{m}{s^2}$ \\
$\mu$:  & Längendichte in $\frac{kg}{m}$ \\
$v$:    & Geschwindigkeit
\end{tabular}
Winkel klein: $\cos\phi \approx 1 \quad \sin\phi \approx \phi \approx \tan\phi$
\[ \boxed{-Fs \sin \phi_L + F_s \sin \phi_R 
= F_s (\frac{\partial}{\partial x} f(x,t) 
- \frac{\partial}{\partial x} f(x-\Delta x,t) )} \]
Wellengleichung: 
\[ \boxed{\frac{F_s}{\mu} \frac{\partial^2}{\partial x^2} f(x,t) 
= \frac{\partial^2}{\partial t^2} f(x,t)} \]
      		% Wellen
% coding:utf-8

%----------------------------------------
%FOSAPHY, a LaTeX-Code for a summary of basic physics
%Copyright (C) 2013, Daniel Winz, Ervin Mazlagic

%This program is free software; you can redistribute it and/or
%modify it under the terms of the GNU General Public License
%as published by the Free Software Foundation; either version 2
%of the License, or (at your option) any later version.

%This program is distributed in the hope that it will be useful,
%but WITHOUT ANY WARRANTY; without even the implied warranty of
%MERCHANTABILITY or FITNESS FOR A PARTICULAR PURPOSE.  See the
%GNU General Public License for more details.
%----------------------------------------

\chapter{Wärme}

\section{Gasgleichung}
\[ \boxed{\frac{p \cdot V}{\vartheta \cdot n} = const} \]
\[ \boxed{p \cdot V = N \cdot k \cdot \vartheta} \]
\[ \boxed{p \cdot V = n \cdot R \cdot \vartheta} \]
$k = 1.38 \cdot 10^{-23} \frac{K}{J}$ \\
$R = 8.31 \frac{J}{mol \cdot K}$ \\
$N$: Anzahl Teilchen \\
$n$: Stoffmenge in Mol \\
$N_A = 6.022 \cdot 10^{23} \frac{1}{mol}$ (Avogadrozahl) \\
\[ n = \frac{N}{N_A} = \frac{m}{M_{mol}} \]
\subsection{Molare Masse}
\[ M_{Luft} = 28.8 \frac{g}{mol} \]

\section{Barometrische Höhenformel}
\[ \boxed{p 
= p_0 \cdot e^{- \frac{M_{mol} \cdot g}{R \cdot \vartheta} \cdot y} 
= p_0 \cdot e^{- \frac{y}{H}}} \qquad \text{$\vartheta$ konstant}\]
\[ \boxed{p = p_0 \cdot \left(1 - \frac{b}{a} \cdot y\right)^
{\frac{M_{mol} \cdot g}{R \cdot b}}} \]
$a = 288.15 K$ \\
$b = 6.5 \frac{K}{km}$ \\

\section{Druck}
\[ \boxed{p = \frac{F}{A} = m_{mol} \cdot {v_x}^2 \cdot \frac{N}{V}} \]

\section{Kinetische Gastheorie}
\[ P = \frac{N}{V} \frac{1}{3} m_{kül}\langle v^2\rangle  = \frac{N}{V} \frac{2}{3} \langle E_{kin, kül}\rangle  = \frac{N}{V} k \vartheta  \]
mikroskopisch: 
\[ \langle E_{kin, kül}\rangle  = \frac{1}{2} m_{kül} \langle v^2\rangle  = 3 \cdot \frac{1}{2} k \vartheta \]
makroskopisch: 
\[ E_{Gas} = 3 \cdot n \frac{1}{2} R \vartheta\]
\[ \langle E_{kin, kül}\rangle  \sim \vartheta \]
\[ \langle E_{kin, kül}\rangle  = 3 \frac{1}{2} k \vartheta \]

\section{Spezifische Wärmekapazität}
\[ Q = m \cdot c \cdot \Delta \vartheta \]
\[ dQ = m \cdot c \cdot d \vartheta \]
\[ c_{(m)} = \frac{1}{m} \frac{dQ}{d \vartheta} \qquad \text{c pro Masse}\]
\[ c_{(n)} = \frac{1}{n} \frac{dQ}{d \vartheta} \qquad \text{c pro Mol}\]

\subsection{Spezifische Wärmekapazität von Gasen}
konstantes Gasvolumen
\[ c_{(n)V} = C_{V} = \frac{\#FG}{2}R \]
              % Wärme


\end{document}

