\chapter{Newtonsche Axiome}
Die Newtonschen Axiome bilden das Fundament der klassischen Mechanik. Diese
hatte Newton im Jahre 1687 in seinem Werk 
\textit{Philosophiae Naturalis Principia Mathematica} formuliert. 

\section{Erstes Newtonsches Axiom}
\textit{"`Ein Körper verharrt im Zustand der Ruhe oder der gleichförmigen 
Translation, sofern er nicht durch einwirkende Kräfte zur Änderung seines 
Zustandes gezwungen wird."' --- Trägheit}

\[ \boxed{\vec{v} = \text{konstant}} \quad \text{falls } \sum\vec{F}=0 \]

\section{Zweites Newtonsches Axiom} 
\textit{"`Die Änderung der Bewegung ist der Einwirkung der bewegenden Kraft 
proportional und geschieht nach der Richtung derjenigen geraden Linie, nach
welcher jene Kraft wirkt."' --- Kraft}

\[ \boxed{\vec{F} = m \cdot \vec{a}} \]

\section{Drittes Newtonsches Axiom}
\textit{"`Kräfte treten immer paarweise auf. Übt ein Körper A auf einen 
anderen Körper B eine Kraft aus (actio), so wirkt eine gleich grosse, aber
entgegen gerichtete Kraft von Körper B auf Körper A (reactio)."' --- 
Aktion und Reaktion}

\[ \boxed{\vec{F}_{A \rightarrow B} = - \vec{F}_{B \rightarrow A}} \]
