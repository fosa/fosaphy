% coding:utf-8

%----------------------------------------
%FOSAPHY, a LaTeX-Code for a summary of basic physics
%Copyright (C) 2013, Daniel Winz, Ervin Mazlagic

%This program is free software; you can redistribute it and/or
%modify it under the terms of the GNU General Public License
%as published by the Free Software Foundation; either version 2
%of the License, or (at your option) any later version.

%This program is distributed in the hope that it will be useful,
%but WITHOUT ANY WARRANTY; without even the implied warranty of
%MERCHANTABILITY or FITNESS FOR A PARTICULAR PURPOSE.  See the
%GNU General Public License for more details.
%----------------------------------------

\section{Reibungskoeffizient}
\begin{footnotesize}
\begin{longtable}{llll}
  \rowcolor{white} \textbf{Material 1} & \textbf{Material 2} 
  & \textbf{Haftreibung $\mu_{HR}$} & \textbf{Gleitreibung $\mu_{GR}$}\\
  \rowcolor{lgray}  Stahl     & Stahl             & 0.74 & 0.57\\
  \rowcolor{white}  Aluminium & Stahl             & 0.61 & 0.47\\
  \rowcolor{lgray}  Kupfer    & Stahl             & 0.53 & 0.36\\
  \rowcolor{white}  Messing   & Stahl             & 0.51 & 0.44\\
  \rowcolor{lgray}  Zink      & Grauguss          & 0.85 & 0.21\\
  \rowcolor{white}  Kupfer    & Grauguss          & 1.05 & 0.29\\
  \rowcolor{lgray}  Glas      & Glas              & 0.94 & 0.40\\
  \rowcolor{white}  Kupfer    & Glas              & 0.68 & 0.53\\
  \rowcolor{lgray}  Teflon    & Teflon            & 0.04 & 0.04\\
  \rowcolor{white}  Teflon    & Stahl             & 0.04 & 0.04\\
  \rowcolor{lgray}  Gummi     & Beton (trocken)   & 1.0  & 0.80\\
  \rowcolor{white}  Gummi     & Beton (nass)      & 0.30 & 0.25
\end{longtable}
\end{footnotesize}