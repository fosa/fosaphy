% coding:utf-8

%----------------------------------------
%FOSAPHY, a LaTeX-Code for a summary of basic physics
%Copyright (C) 2013, Daniel Winz, Ervin Mazlagic

%This program is free software; you can redistribute it and/or
%modify it under the terms of the GNU General Public License
%as published by the Free Software Foundation; either version 2
%of the License, or (at your option) any later version.

%This program is distributed in the hope that it will be useful,
%but WITHOUT ANY WARRANTY; without even the implied warranty of
%MERCHANTABILITY or FITNESS FOR A PARTICULAR PURPOSE.  See the
%GNU General Public License for more details.
%----------------------------------------

\chapter{Schwingung}
\section{Begriffe}
\begin{tabular}{ll}
$\omega:$       & Kreisfrequenz \\
$f:$            & Frequenz \\
$T:$            & Periodendauer \\
$A:$            & Amplitude \\
$\phi:$         & Phasenverschiebung \\
$x:$            & Auslenkung \\
$\theta:$       & Auslenkwinkel \\
$I:$            & Trägheitsmoment $[kg m^2]$ \\
$k:$            & Federkonstante $[\frac{N}{m}]$ \\
$\kappa:$       & Rotationsfederkonstante $[Nm]$ \\
$d:$            & Abstand Massenschwerpunkt zu Drehachse \\
$L:$            & Pendellänge \\
$\ell:$         & Länge der Flüssigkeitssäule \\
$\beta:$        & Abklingkonstante $[s^{-1}]$ \\
$b:$            & Dämpfungskonstante $[\frac{kg}{s}]$ \\
$\tau:$         & Zeitkonstante, Zerfallszeit $[s]$ \\
$Q:$            & Güte \\
$H:$            & Erregerauslenkung \\
$\Omega_R:$     & Resonanzkreisfrequenz \\
$\Delta\Omega:$ & Kurvenbreite, Bandbreite \\
\end{tabular}

\section{Einfache harmonische Schwingung}
Differentialgleichung: 
\[ \boxed{F = m \cdot \ddot{x} + k \cdot x = 0} \]
\[ \boxed{x(t) = A \cdot \cos(\omega \cdot t + \phi)} \]
\[ \boxed{\dot{x}(t) = - \omega A \cdot \sin(\omega \cdot t + \phi)} \]
\[ \boxed{\ddot{x}(t) = - \omega^2 A \cdot \cos(\omega \cdot t + \phi)} \]
\[ \boxed{\omega = \sqrt{\frac{k}{m}}} \]
\[ \boxed{f = \frac{\sqrt{\dfrac{k}{m}}}{2 \pi}} \]
\[ \boxed{T = 2 \cdot \pi \sqrt{\frac{m}{k}}} \]

\subsection{Auslenkung und Geschwindigkeit als Anfangsbedingungen}
\[ \boxed{x_0 = x(0) = A \cdot \cos(\phi)} \]
\[ \boxed{v_0 = \dot{x}(0) = - \omega \cdot A \cdot \sin(\phi)} \]
\[ \boxed{\phi = \tan^{-1}\left(-\frac{v_0}{\omega \cdot x_0}\right)} \]
\textbf{Achtung!} Lösung kann im falschen Quadranten liegen. 
Evtl. mit $pi$ addieren. 
\[ \boxed{A = \sqrt{{x_0}^2 + \left(\frac{v_0}{\omega}\right)^2}} \]

\subsection{Geschwindigkeit und Beschleunigung aus Amplitude und Position}
\[ \boxed{v(t) = \sqrt{\frac{k}{m}} \cdot \sqrt{A^2 - x^2(t)}} \]
\[ \boxed{a(t) = - \frac{k}{m} \cdot x(t)} \]

\subsection{Maximale Geschwindigkeit und Beschleunigung}
\[ \boxed{v_{max} = \omega \cdot A} \]
\[ \boxed{v_{min} = -\omega \cdot A} \]
\[ \boxed{a_{max} = -\omega^2 \cdot A} \]

\subsection{Vertikales Federpendel}
\[ \boxed{\omega = \sqrt{\frac{k}{m}}} \]
\[ \boxed{x(t) = - \frac{m \cdot g}{k} + A \cdot \cos(\omega \cdot t + \phi)} \]

\subsection{Energie}
\[ \boxed{E_{pot}(t) = \frac{1}{2} \cdot k \cdot x(t)^2} \]
\[ \boxed{E_{kin}(t) = \frac{1}{2} \cdot m \cdot v(t)^2} \]
\[ \boxed{E_{tot} = \frac{1}{2} \cdot k \cdot A^2 
= \frac{1}{2} \cdot m \cdot \omega^2 \cdot A^2 
= \frac{1}{2} \cdot m \cdot (v_{max})^2} \]

\section{Rotationsschwingung}
Differentialgleichung: 
\[ \boxed{M = I_s \cdot \ddot{\theta} + \kappa \cdot \theta = 0} \]
\[ \boxed{\theta(t) = \theta_{max} \cdot \cos(\omega \cdot t + \phi)} \]
für kleine Winkel: 
\[ \boxed{\kappa = k_1 \cdot {r_1}^2 + k_2 \cdot {r_2}^2 + \dots } \qquad 
\text{$r$: Abstand Feder zu Drehpunkt} \]
\[ \boxed{\kappa = \sum_{1}^{n} \left(k_n \cdot {r_n}^2\right)} \]
\[ \boxed{\omega = \sqrt{\frac{\kappa}{I}}} \]
\[ \boxed{f = \frac{\sqrt{\dfrac{\kappa}{I}}}{2 \pi}} \]
\[ \boxed{T = 2 \cdot \pi \cdot \sqrt{\frac{I}{\kappa}}} \]

\section{Physikalisches Pendel}
Differentialgleichung: 
\[ \boxed{M = I_z \cdot \ddot{\theta} + d \cdot m \cdot g \cdot \sin(\theta) = 0} \]
für kleine Winkel: 
\[ \boxed{M = I_z \cdot \ddot{\theta} + d \cdot m \cdot g \cdot \theta = 0} \]
\[ \boxed{\kappa = d \cdot m \cdot g} \qquad 
\text{Ist der Schwerpunkt über dem Drehpunkt, ist $\kappa$ negativ} \]
\[ \boxed{\kappa_{tot} = \kappa_1 + \kappa_2 \dots} \]
\[ \boxed{\kappa_{tot} = \sum_{1}^{n} \left(d_n \cdot m_n \cdot g\right) } \]
\[ \boxed{\omega = \sqrt{\frac{\kappa}{I_z}} 
= \sqrt{\frac{m \cdot g \cdot d}{I_z}}} \]
\[ \boxed{f = \frac{\sqrt{\frac{\kappa}{I_z}}}{2 \pi}
= \frac{\sqrt{\frac{m \cdot g \cdot d}{I_z}}}{2 \pi}} \]
\[ \boxed{T = 2 \cdot \pi \cdot \sqrt{\frac{I_z}{m \cdot g \cdot d}}} \]
speziell: Fadenpendel
\[ \boxed{I = m \cdot L^2} \]
\[ \boxed{\kappa = m \cdot g \cdot L} \]
\[ \boxed{\omega = \sqrt{\frac{g}{L}}} \]
\[ \boxed{f = \frac{\sqrt{\frac{g}{L}}}{2 \cdot \pi}} \]
\[ \boxed{T = 2 \cdot \pi \cdot \sqrt{\frac{L}{g}}} \]
Fadenkraft: 
\[ \boxed{F_s = m \cdot g \cdot \cos(\theta) + m \cdot \frac{v^2}{L} 
= m \cdot g \cdot \cos(\theta) + m \cdot \omega^2 \cdot L} \]

\section{Pendelnde Flüssigkeitssäule}
\[ \boxed{\omega = \sqrt{\frac{2 \cdot g}{\ell}}} \]
\[ \boxed{f = \frac{\sqrt{\frac{2 \cdot g}{\ell}}}{2 \pi}} \]
\[ \boxed{T = 2 \pi \sqrt{\frac{\ell}{2 \cdot g}}} \]

\section{Gedämpfte Schwingung}
Differentialgleichung: 
\[ \boxed{F = m \cdot \ddot{x} + \underbrace{b \cdot \dot{x}}_
{\text{Dämpfung}} + m \cdot x = 0} \]
\[ \boxed{F_{\text{Dämpfung}} = F_{\text{Stokes}} 
= 6 \pi \cdot \eta \cdot R \cdot v} \]
\[ \boxed{\rightarrow b = 6 \pi \cdot \eta \cdot R} \]
\[ \boxed{\omega = \sqrt{\frac{k}{m}}} \]
\[ \boxed{\beta = \frac{b}{2 m}} \]
Fall 1: $\beta > \omega$ "'Kriechfall"'
\[ \boxed{x(t) \sim e^{(-\beta \pm \delta)t}} \]
Fall 2: $\beta = \omega$ "'kritische Dämpfung"'
\[ \boxed{b = b_{krit} = \sqrt{4 \cdot k \cdot m}} \]
Fall 3: $\beta < \omega$ "'Gedämpfte Schwingung"'
\[ \boxed{x(t) = A \cdot e^{-\beta t} \cdot \cos(\omega_d \cdot t)} \]
\[ \boxed{\omega_d = \sqrt{\omega^2 - \beta^2} 
= \sqrt{\frac{k}{m} - \left({\frac{b}{2 \cdot m}}\right)^2}} \]
Zerfallszeit: 
\[ \boxed{\tau = \frac{1}{\beta}} \]
\[ \boxed{A(t) = A \cdot e^{-\beta t} = A \cdot e^{-\frac{t}{\tau}}} \]
Bei der Zeit $t = \tau$
\[ \boxed{A(\tau) = \frac{A_0}{e} = A_0 \cdot e^{-1} \approx 0.37 \cdot A_0} \]
Abklingkonstante: 
\[ \boxed{\beta 
= \frac{\ln\left(\frac{x(t_1)}{x(t_2)}\right)}{t_2 - t_1}} \]
Schwingungsenergie: 
\[ \boxed{E(t) = \frac{m}{2} \cdot \dot{x}^2(t) + \frac{k}{2} \cdot x^2(t)} \]
Mittlere Energie: 
\[ \boxed{\langle E(t)\rangle = E_0 \cdot e^{-\frac{2 \cdot t}{\tau}}} \]
Güte: 
\[ \begin{array}{l}
\boxed{Q = \frac{2 \pi \cdot E(t)}{|\Delta E(t)|} 
= \frac{\pi}{\beta \cdot T_d} = \frac{\pi \cdot \tau}{T_d} 
= \frac{\omega_d \cdot \tau}{2}} \\
\text{$\Delta E(t)$: Energieverlust pro Periode}
\end{array} \]

\section{Erzwungene Schwingung}
\[ \boxed{x(t) = A(\Omega) \cdot \cos(\Omega t - \varphi(\Omega))} \]
\[ \boxed{A(\Omega) = \frac{F_0}{m \cdot 
\sqrt{(\omega^2 - \Omega^2)^2 + (2 \cdot \Omega \cdot \beta)^2}} 
\approx \frac{H}{\sqrt{\left(1 - \left(\frac{\Omega}{\omega}\right)^2\right)^2 
+ \left(\frac{\Omega}{Q \cdot \omega}\right)^2}}} \]  
\[ F_0\text{: Anregungskraft} \qquad H\text{: Anregungsamplitude} \]
\[ \boxed{\varphi(\Omega) = \arctan\left(\frac{2 \cdot \Omega \cdot \beta}
{{\omega}^2 - \Omega^2}\right) 
= \arctan{\left(\frac{\frac{b}{\sqrt{k \cdot m}} 
\left(\frac{\Omega}{\omega}\right)}
{1 - \left(\frac{\Omega}{\omega}\right)^2}\right)}} \]
\[ \boxed{H = \frac{F_0}{k}} \]
\[ \boxed{\omega = \sqrt{\frac{k}{m}}} \]
\[ \boxed{\beta = \frac{b}{2 m} = \frac{1}{\tau}} \]
\[ \boxed{Q = \frac{\omega_d}{2\beta} = \omega_d \frac{m}{b} 
= \pi \frac{\tau}{T}} \]
schwache Dämpfung: 
\[ \boxed{b << b_{krit} = \sqrt{4 k \cdot m} \quad Q >> 1} \]

\subsection{Resonanzfrequenz}
\[ \boxed{\Omega_R = \sqrt{\omega^2 - 2\beta^2} \neq \omega_d 
= \sqrt{\omega^2 - \beta^2} \neq \omega} \]
\[ \boxed{A(\Omega_R) = \frac{\omega^2 H}{2 \beta \sqrt{\omega^2 - \beta^2}} 
= \frac{k H}{m \sqrt{(\omega^2 - {\Omega_R}^2)^2 + (2\beta\Omega_R)^2}} 
\approx \frac{Q H}{\sqrt{1 - \frac{1}{2 Q^2}}} \approx Q \cdot H} \]
\[ \boxed{\varphi_R = \varphi(\Omega_R) 
= \arctan\left(\frac{2 \beta \omega_R}{\omega^2 - {\Omega_R}^2}\right) 
\approx \arctan\left(\sqrt{4 Q^2 - 2}\right) \approx \frac{\pi}{2}} \]
Für $Q>5$
\[ \boxed{\Omega_R \approx \omega \left(1 - \frac{1}{4Q^2}\right)} \]
\[ \boxed{A_R = A(\Omega_R) \approx \frac{Q H}{\sqrt{1 - \frac{1}{2Q^2}}} 
\approx Q \cdot H} \]
\[ \boxed{Q \approx \frac{\Omega_R}{\Delta\Omega}} \]
