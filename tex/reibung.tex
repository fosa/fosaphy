% coding:utf-8

%----------------------------------------
%FOSAPHY, a LaTeX-Code for a summary of basic physics
%Copyright (C) 2013, Daniel Winz, Ervin Mazlagic

%This program is free software; you can redistribute it and/or
%modify it under the terms of the GNU General Public License
%as published by the Free Software Foundation; either version 2
%of the License, or (at your option) any later version.

%This program is distributed in the hope that it will be useful,
%but WITHOUT ANY WARRANTY; without even the implied warranty of
%MERCHANTABILITY or FITNESS FOR A PARTICULAR PURPOSE.  See the
%GNU General Public License for more details.
%----------------------------------------

\chapter{Reibung}

Die Reibung bezeichnet in der Pyhsik eine Eigenschaft, welche stets einer
Bewegung entgegenwirkt. In der Mechanik kann sie als Kraft verstanden
werden, welche zwischen sich berührenden Körpern auftritt. Hiebei gibt es
einige Arten der Reibung, die aufgrund ihres speziellen Charakters, 
unterschieden werden. Die wohl wichtigsten sind die Haft- und 
Gleitreibung (weiter gibt es z.B. noch die Roll-, Wälz-, Bohrreibung).

\newpage
\section{Reibungskraft}
Die Reibungskraft $\vec{F}_R$ ist eine Kraft welche zwischen sich
berührenden Körpern auftritt. Sie ist definiert als das Produkt aus
der Kraft $\vec{F}_N$ welche normal zwischen den Körpern anliegt
und dem Reibungskoeffizienten $\mu$ welcher die 
Oberflächenbeschaffenheit beschreibt. 

\begin{figure}[h!]
	\centering
	\includegraphics[scale=0.8]{reibung.pdf}
	\caption{Reibungskraft für eine Masse die auf einer Unterlage 
		liegt mit $\sum \vec{F}_{extern}$ positiv (l)
		und negativ (r).}
\end{figure}

\noindent
Die Richtung der Reibungskraft ist dabei stets entgegen der 
anliegenden äusseren Nettokraft 
$\sum \vec{F}_{extern}$.

\[ \boxed{
	\vec{F}_R = \vec{F}_N \cdot \mu
		\qquad ,\vec{F}_R \bot \vec{F}_N
}\]

\section{Bewegungsbedingung}
\noindent
Ein Körper der eine Reibung kennt, bringt eine Kraft $\vec{F}_R$ 
auf entgegen der anliegenden Nettokraft $\sum \vec{F}_{extern}$. 
Das bedeutet, dass ein solcher Körper in Ruhe bleibt, solange die 
anliegende Nettokraft $\sum \vec{F}_{extern} \leq \vec{F}_R$ ist. 
Somit ergibt sich die Bewegungsbedingung zu

\[ \boxed{\begin{array}{l r l}
	\text{Körper bleibt in Ruhe falls } & 
		\sum \vec{F}_{extern} & \leq \vec{F}_R \\
	& & \\
	\text{Körper bewegt sich falls } &
		\sum \vec{F}_{extern} & > \vec{F}_R 
\end{array}} \]

\section{Reibungskoeffizient}
Der Reibungskoeffizient $\mu$ wird typischerweise für zwei Fälle 
angegeben, welche voneinander zu unterscheiden sind.
\begin{itemize}
	\item Haften $\Rightarrow$ Haftreibungskoeffizient $\mu_{HR}$
	\item Gleiten $\Rightarrow$ Gleitreibungskoeffizient $\mu_{GR}$
\end{itemize}

\noindent
Für den Reibungskoeffizienten gilt stets, dass der Koeffizient für die
Haftung grösser oder zumindest gleich ist wie für das Gleiten 
(bei den selben Bedingungen).

\[ \boxed{
	\mu_{HR} \geq \mu_{GR}
}\]

\begin{footnotesize}
\begin{longtable}{llll}
  \rowcolor{white} \textbf{Material 1} & \textbf{Material 2} 
  & \textbf{Haftreibung $\mu_{HR}$} & \textbf{Gleitreibung $\mu_{GR}$}\\
  \rowcolor{lgray}  Stahl     & Stahl             & 0.74 & 0.57\\
  \rowcolor{white}  Aluminium & Stahl             & 0.61 & 0.47\\
  \rowcolor{lgray}  Kupfer    & Stahl             & 0.53 & 0.36\\
  \rowcolor{white}  Messing   & Stahl             & 0.51 & 0.44\\
  \rowcolor{lgray}  Zink      & Grauguss          & 0.85 & 0.21\\
  \rowcolor{white}  Kupfer    & Grauguss          & 1.05 & 0.29\\
  \rowcolor{lgray}  Glas      & Glas              & 0.94 & 0.40\\
  \rowcolor{white}  Kupfer    & Glas              & 0.68 & 0.53\\
  \rowcolor{lgray}  Teflon    & Teflon            & 0.04 & 0.04\\
  \rowcolor{white}  Teflon    & Stahl             & 0.04 & 0.04\\
  \rowcolor{lgray}  Gummi     & Beton (trocken)   & 1.0  & 0.80\\
  \rowcolor{white}  Gummi     & Beton (nass)      & 0.30 & 0.25
\end{longtable}
\end{footnotesize}

